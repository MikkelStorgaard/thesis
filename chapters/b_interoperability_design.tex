\chapter{\fshark{}s interoperability between \fsharp and Futhark (\csharp{})}
FShark stands on three legs: The FShark compiler, the Futhark C\# code generator, and
the FSharkWrapper.
The compiler is responsible for compiling FShark source code into Futhark
source code, and the C\# code generator takes the result Futhark source code,
and compiles OpenCL powered C\# libraries, which can be imported directly back
into F\#.

It is of course possible to use the compiler and the code generator as
individual modules, but for this project, the FSharkWrapper has been designed to
let users use FShark without having to understand any of the underlying
pipeline.

To illustrate this; take a look at figure \ref{fig:fsharkusingwrapper}. In the
first line, the user initializes the FSharkWrapper with the arguments necessary
to use the wrapper itself. In the second line, the user adds a source file to
the wrapper by it's path.
In the third line, the user tells the wrapper to run the compilation pipeline.
Assuming that the compilation goes well, the user can then invoke some function
from the FShark program in line four.

Here, calling the \texttt{CompileAndLoad()} function triggers the entire \fshark{}
pipeline as described in \ref{pipeline}, and does then have a function available
for the user to call afterwards.

However, as this is the default way of using \fshark{}, we are currently calling
\texttt{CompileAndLoad()} every time we use the \fshark{} program.
This is happening even though we only need the final compiled C\# assembly to
load back into \fsharp{} at runtime. 

Everytime we run the FShark compiler pipeline, we are therefore also
\begin{enumerate}
\item parsing, typechecking and generating a TAST from the \fshark{} code,
  using FSharp's compiler.
\item generating Futhark source code from the FSharp TAST
\item Writing the Futhark source code to disk
\item running the Futhark compiler and C\# code generator on the Futhark source
  code
\item running the mono C\# compiler on the resulting C\# source code
\end{enumerate}

For two selected benchmarks we have the following times
% Numbers for nbody.fut:
%  FShark parsing took 1850345 ms
%  FSharpDecls to FSharkIL took 76660 ms
%  FSharkIL to Futhark source code took 43471 ms
%  Compiling the Futhark module into .cs source code took 1555105 ms
%  Compiling .cs module took 999251 ms
%  Loading compiled .cs assembly using reflection took 101601 ms
%  The entire FShark compilation pipeline took 4634460 ms
%  Average invokation time was 104227 ms
%  Invoking main took 107597 ms

% LocVolCalib.fut small
%FShark parsing took 217984 ms
%FSharpDecls to FSharkIL took 19129 ms
%FSharkIL to Futhark source code took 98949 ms
%Compiling the Futhark module into .cs source code took 8037165 ms
%Compiling .cs module took 1422527 ms
%Loading compiled .cs assembly using reflection took 81370 ms
%The entire FShark compilation pipeline took 9877907 ms
%Average invokation time was 219150 ms
%
% locvolcalib medium
% FShark parsing took 221395 ms
% FSharpDecls to FSharkIL took 21487 ms
% FSharkIL to Futhark source code took 107558 ms
% Compiling the Futhark module into .cs source code took 7739211 ms
% Compiling .cs module took 1467667 ms
% Loading compiled .cs assembly using reflection took 77850 ms
% The entire FShark compilation pipeline took 9635733 ms
% Average invokation time was 333960 ms
% 
% locvolcalib large
% FShark parsing took 229771 ms
% FSharpDecls to FSharkIL took 20973 ms
% FSharkIL to Futhark source code took 111219 ms
% Compiling the Futhark module into .cs source code took 7452111 ms
% Compiling .cs module took 1422789 ms
% Loading compiled .cs assembly using reflection took 79751 ms
% The entire FShark compilation pipeline took 9317354 ms
% Average invokation time was 5863264 ms

% nbody from assembly:
% Opening class from assembly took 61841 ms
% Running nbody from assembly took 117373 ms

% locvolcalib
% small
% Opening class from assembly took 147341 ms
% Running locvol small from assembly took 210698 ms
% medium
% Opening class from assembly took 61771 ms
% Running locvol medium from assembly took 314444 ms
% large
% Opening class from assembly took 71198 ms




% make a chart that shows the time consumption of a regular FShark execution
% (for LocVolCalib and nbody) from start to finish, and what the time is spent
% on 

% make the same chart but where the assembly is referenced at compile time,
% which means that the FShark program is just used as a compiler.

% for both tests, we use ArrayToFlatArray, as we have still settled on normal
% arrays as the array.

\subsection{Pros and cons of the current design}
As there are demonstratively great performance gains to be won by only using the
compiler part of the FShark pipeline, it is worth discussing whether the
rest of the FShark pipeline should remain.

Besides eliminating the entire compilation operation at every \fshark{}
execution, a compiler-only approach to the \fshark{} compiler would give us the
following advantages:
\begin{itemize}
\item \textbf{Standalone-modules first:} As the compiler is now only
  used once, the resulting Futhark assembly is readily importable in any .NET
  project, as long as the required Mono libraries are also available.
  This goes not only for the user who just compiled the assembly, but also for
  any other user who has acquired the necessary Mono libraries. This means that
  the \fshark{} developer can use and share the \fshark{} assemblies with
  colleagues and coworkers like any other sharable .NET library,
  as this is indeed what a compiled Futhark
  \csharp{} library amounts to.

  Corollarily; this \fshark{} design would make \fshark{} is useful for generating
  high-performing .NET libraries. (Although one could write such libraries in
  Futhark instead of \fshark{}.)

\item \textbf{Static typing of \fshark{} module:} The current runtime-only
  approach means that the user must rely on reflection to call \fshark{}
  functions.
  In this situation, all modern IDE comforts like autocompletion, and especially
  static type checking and inference falls away.
  For the following example
  %% let foo (x : int) ((y,z) : (int * int)) : (int * int) 
  , the current design demands that we first wrap our arguments in an
  \texttt{obj array}, before invoking the function \texttt{foo}. Furthermore, we
  must also downcast the result using \fsharp{}s downcasting operator \texttt{:?>}.
  Because we are upcasting our arguments to an \texttt{obj array}, we can
  actually pass any (correctly casted) array as an argument to our
  reflection-invoked function, without triggering any type errors at compile
  time.
  The same goes for the downcasted result from the function. We can cast the
  result as whatever type we like, and not run into any trouble until we finally
  run the compiled program.
  % write the example.
  However, if we use \fshark{} to generate assemblies instead, we now have all
  the type information available at compile time. Our compiler will block us
  from compiling the program by giving us useful type errors. Last but not
  least, we can remove all the casting operations that are littering the
  program.
  % other example
  
\item \textbf{Getting rid of, or at least trimming down, the FSharkWrapper:} 
\end{itemize}

However, the current design of \fshark{} also has some advantages that follows
automatically from the design.

\begin{itemize}
\item \textbf{Rapid \fshark{} code development:}
  Currently, it is recommended that any \fshark{} code for a project is also
  built as part of the project.
  By including the \fshark{} file in the original project's source list, we can
  call the \fshark{} module natively, without running the \fshark{} compiler, to
  prototype and debug the \fshark{} code directly in our IDE, before we switch
  to using the compiled version of the \fshark{} code.
  % example of developing
  % 1: picture of placeholder function from Fshark, and fsharp module calling this
  % 2: picture of placeholder function filled out, and fsharp module calling
  % this
  % 3: picture of placeholder function filled out, and fsharp module calling
  % this through fshark

\item \textbf{En mere}
\end{itemize}


% Discuss the pros and cons of the current design:
%% Pros:
%% 1) rapid development. FShark development can take place within the same IDE
%% and project as the FShark-using program.
%% 2) Ease of use: FShark can be used through a few imports and 

%% Cons:
%% 1) Probably the time consumption
%% 2) The need for the FShark wrapper. Because the runtime-only design demands
%% using reflection, the reflection behaviour has been encapsulated in a bulky
%% wrapper object.
\section{The future of \fshark{} interoperability}
With these considerations in mind, my future work on \fshark{}s interoperability
consists of reducing FSharkWrapper in size, so it only takes an \fshark{} source path
  and a .NET assembly outpath as inputs, and does nothing more than
  orchestrating the \fshark{}-, the Futhark- and the \csharp compiler.
  The current design is too complex, largely from supporting too many superflous
  features like concatenating multiple sources, and so on.

I will also be researching the optimal way to keeping the \fshark{} module development as
  close to the rest of the \fshark{}-using project as possible, without %% MERE
  %% HERE
  
The current design enables direct prototyping, which must of course be kept
in later versions of \fshark{}.

%% Another idea: keep annoying wrapper, but let it skip the compiling step.

% \fshark{} as a library generator

%%% Local Variables:
%%% mode: latex
%%% TeX-master: "../thesis"
%%% End:


