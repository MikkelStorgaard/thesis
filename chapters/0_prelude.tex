\maketitle{}

\begin{abstract}
  Futhark is a purely functional programming language, designed to be compiled
  to efficient GPU code.
  The GPU code has both C and Python interfaces, but does not interoperate with
  larger mainstream languages.

  We first describe and implement a \csharp{} code generator that allows GPU kernels
  written in Futhark to be compiled and used as external libraries in \csharp{}-
  and \fsharp{} programs.

  We then describe and implement the language \fshark{}, which lets developers
  write and prototype their own GPU code entirely as declarative \fsharp{} code.
  The language comes with a compiler of its own, which compiles \fshark{}
  programs to Futhark code.

  We argue that the \fshark{}-to-Futhark translations are correct by comparing
  the results of \fsharp{}-evaluated \fshark{} code with their Futhark-evaluated
  counterparts.

  We show that Futhark can be compiled to \csharp{} GPU code without significant
  ($\pm 1\%$)
  runtime reductions (speedups) compared to C- and Python GPU code,
  by comparing runtimes across over 30 Futhark benchmark programs ported from other array
  programming.

  Finally, we show that complex GPU benchmarks can be
  succesfully ported from Futhark to \fshark{} without significant rewriting.
  The benchmarks are compiled to GPU code and executed in an \fsharp{} context
  without significant speed decreases compared to the original Futhark versions.
\end{abstract}

\clearpage
 
\tableofcontents

\clearpage

\chapter*{Preface}
This thesis is submitted in fulfillment of a 30 ECTS master's thesis in Computer
Science (Datalogi) at the University of Copenhagen, for Mikkel Storgaard Knudsen.

(Front page picture: The humble hedgehog is a peaceful omnivore firmly placed in
the lower-middle part of the food chain. However, sneaky disguises occasionally allows it
to partake in feasts that are otherwise reserved for apex predators.)

%%% Local Variables:
%%% mode: latex
%%% TeX-master: "../thesis"
%%% End: