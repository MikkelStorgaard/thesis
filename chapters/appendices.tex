\begin{appendices}
\chapter{\fshark{} standard library}
\label{appendix:soacs}




\chapter{Program for benchmarking byte memory writes in \csharp{}}
\begin{minted}[linenos, breaklines]{csharp}
using System;
using System.Diagnostics;
using System.Runtime.InteropServices;

namespace ConsoleApplication2
{
    internal class Program
    {
        static private int TEST_SIZE = 1000000;
        
        static void UsingBuffer()
        {
            byte[] target = new byte[TEST_SIZE*sizeof(int)];
            for (int i = 0; i < TEST_SIZE; i++)
            {
                var intAsBytes = BitConverter.GetBytes(i);
                Buffer.BlockCopy(intAsBytes, 0, target, i * sizeof(int), sizeof(int)); 
            }
        }
        
        static void UsingUnsafe1()
        {
            byte[] target = new byte[TEST_SIZE*sizeof(int)];
            for (int i = 0; i < TEST_SIZE; i++)
            {
                unsafe
                {
                    fixed (byte* ptr = &target[i * sizeof(int)])
                    {
                        *(int*) ptr = i;
                    }
                }
            }
        }
        
        static void UsingUnsafe2()
        {
            byte[] target = new byte[TEST_SIZE*sizeof(int)];
            unsafe
            {
                fixed (byte* ptr = &target[0])
                {
                    for (int i = 0; i < TEST_SIZE; i++)
                    {
                        *(int*) (ptr+i*sizeof(int)) = i;
                    }
                }
            }
        }

        public static void Main(string[] args)
        {
            var TESTS = 10;
            var stopwatch = new Stopwatch();
            for (int i = 0; i < TESTS; i++)
            {
                stopwatch.Start();
                UsingBuffer();
                stopwatch.Stop();
            }

            Console.WriteLine("Safe took {0} ticks on avg.", stopwatch.ElapsedTicks / 10);

            stopwatch.Reset();

            for (int i = 0; i < TESTS; i++)
            {
                stopwatch.Start();
                UsingUnsafe1();
                stopwatch.Stop();
            }

        Console.WriteLine("Unsafe1 took {0} ticks on avg.", stopwatch.ElapsedTicks / 10);
            
            stopwatch.Reset();
            
            for (int i = 0; i < TESTS; i++)
            {
            stopwatch.Start();
            UsingUnsafe2();
            stopwatch.Stop();
            }
                
            Console.WriteLine("Unsafe2 took {0} ticks on avg.", stopwatch.ElapsedTicks / 10);

        }
    }
}
\end{minted}
{Short \csharp{} program that measures performance differences between
  various methods of writing scalars to byte arrays}
\label{fig:memoryperformancebenchmark}

\chapter{LocVolCalib benchmark written in \fshark{}}
\label{app:fsharklocvolcalib}
\chapter{LocVolCalib benchmark written in Futhark}
\label{app:futharklocvolcalib}

\chapter{nbody benchmark written in \fshark{}}
\label{app:fsharknbody}
\chapter{nbody benchmark written in Futhark}
\label{app:futharknbody}












\chapter{Numbers for LocVolCalib benchmarks}




\subsubsection{\fshark{} LocVolCalib with OpenCL (avg. of 10 runs)}
Runtimes for compiled \fshark{} version of LocVolCalib with OpenCL enabled.\\
\begin{tabular}{|c|c|c|}
  \hline
  \textbf{small.in} & \texttt{medium.in} & \texttt{large.in}\\ \hline \hline
187972 $\mu s$& 269265 $\mu s$ &5052080 $\mu s$\\ \hline
\end{tabular}

\subsubsection{\fshark{} LocVolCalib without OpenCL (avg. of 10 runs)}
Runtimes for compiled \fshark{} version of LocVolCalib with OpenCL disabled.\\
\begin{tabular}{|c|c|c|}
  \hline
  \textbf{small.in} & \texttt{medium.in} & \texttt{large.in}\\ \hline \hline
79265000 $\mu s$& 154488000 $\mu s$ & 2448660 $\mu s$\\ \hline
\end{tabular}

\subsubsection{Futhark benchmark version of LocVolCalib with \csharp{} OpenCL (avg. of 10 runs)}
These benchmarks are for the futhark-benchmarks version of the LocVolCalib
benchmark, compiled with the \csharp{} OpenCL code generator.
\begin{tabular}{|c|c|c|}
  \hline
  \textbf{small.in} & \texttt{medium.in} & \texttt{large.in}\\ \hline \hline
134031 $\mu s$& 134761 $\mu s$ & 1925001 $\mu s$\\ \hline
\end{tabular}

\subsubsection{Futhark benchmark version of LocVolCalib with \C{} OpenCL (avg. of 10 runs)}
These benchmarks are for the futhark-benchmarks version of the LocVolCalib
benchmark, compiled with the existing \C{} OpenCL code generator.
\begin{tabular}{|c|c|c|}
  \hline
  \textbf{small.in} & \texttt{medium.in} & \texttt{large.in}\\ \hline \hline
135190 $\mu s$& 134428 $\mu s$ & 1911079 $\mu s$\\ \hline
\end{tabular}
\chapter{nbody benchmark written in \fshark{} and Futhark}
The contents...
\end{appendices}






%%% Local Variables:
%%% mode: latex
%%% TeX-master: "../thesis"
%%% End: