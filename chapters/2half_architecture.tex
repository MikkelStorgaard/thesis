\chapter{Architecture and use cases}
\section{Futhark-to-\csharp{}}
Initially, we are interested in making Futhark-generated GPU kernels available
in \csharp{} programs. Specifically, we want to be able to follow use cases such
as this:
\begin{enumerate}
\item We write a short Futhark program, which has a single entry function
  available. This program takes an array of integers, and
adds 2 to each element in the array. The program is shown in figure \ref{fig:shortfutharkprogram0}. 
\item We then compile the Futhark program into a library file, by calling the
  Futhark compiler from the command line, like shown in figure \ref{fig:shortfutharkprogram1}.
This compiles \mathtt{mapPlus2.fut} to a \csharp{} file called MapPlus2.cs

\item Finally, we write a short \csharp{} program in which we want to integrate
  the mapPlus2 function in our program. Such a program is shown in figure \ref{fig:shortfutharkprogram2}.
  In this program, we are importing the Futhark library on line N and constructing
  an instance of the contained Futhark class on line N.\\
  On line N, we generate an array of integers from 0 to 1000000, and finally
  on line N, we use the exposed Futhark function mapPlus2 to add 2 to every
  element in our array.
\end{enumerate}
\begin{figure}[H]
  \centering
  \begin{lstlisting}[language=Futhark]
entry mapPlus2 (xs : []i32) : []i32 =
  map (+2) xs
  \end{lstlisting}
  \caption{A short Futhark program called mapPlus2.fut}
  \label{fig:shortfutharkprogram0}
\end{figure}

\begin{figure}[H]
  \centering
  \begin{lstlisting}[language=bash]
$ futhark-cs --library -o MapPlus2.cs mapPlus2.fut
  \end{lstlisting}
  \caption{We call the Futhark-to-\csharp{} compiler \texttt{futhark-cs} on
    mapPlus2.fut}
  \label{fig:shortfutharkprogram1}
\end{figure}

\begin{figure}[H]
  \centering
\begin{minted}[linenos]{csharp}
using System.Linq;
using MapPlus2;

public class Program
{
    public static int Main(string[] args)
    {
        var mapplus2Class = new MapPlus2();
        var xs = Enumerable.Range(0, 1000000).ToArray();
        var xs_result = mapplus2Class.mapPlus2(xs)
    }
}
\end{minted}
  \caption{We use the compiled Futhark program as any other library.}
  \label{fig:shortfutharkprogram2}
\end{figure}

We want to 
To use this 
%%% Local Variables:
%%% mode: latex
%%% TeX-master: "../thesis"
%%% End:
