\chapter{Introduction}

{\bf High-Level Problem.} 
Developers worldwide are, and have always been, on the lookout 
for increased computing performance.
Until recently, the increased performance could easily be
achieved through advances within raw computing power, as CPU's had steadily been
doubling their number of on-chip transistors, in rough accordance to Moore's Law (citér
her).

Since increasing the frequency of the single CPU has hit the power
wall\cite{powerwall} (among other things), achieving higher performance 
is realized nowadays by scaling the parallelism of hardware, for example
leading to multi-core and then to many-core architectures.
%, where each core provides support for hardware multi-threading.
As the number of cores increases, so does the number of
active threads available for parallel data processing.

Modern mainstream (general-purpose) graphical processing units (GPUs) can run 
tens of thousands of hardware threads in parallel. Modern mainstream CPUs, 
like the current Ryzen series by AMD, usually support between 10 and 20 
simultaneous threads. This makes GPUs an attractive target for data-parallel 
programming.

While re-writing a program for multi-core execution is not without challenges,
GPU programming is significantly more difficult. For example, the typical
approach for multicores is to exploit (only) the parallelism of the outermost
(parallel) loop -- this is because, the loop count is typically greater than 
the number of cores (maximum $32$), which means that the hardware
is fully utilized. GPUs, on the other hand, expose a massive amount of hardware
parallelism, whose full utilization often requires to exploit multiple (nested)
levels of the application's parallelism. Re-structuring the program in this
manner is notoriously difficult, not only in terms of tedious work, but also
because it requires specialized knowledge of compiler analysis.

To make matters even more unattractive, the mainstream APIs for programming
GPUs (e.g., CUDA, OpenCL) are quite low-level and verbose.  For example, GPU
developers must not only write the computational kernels for the GPUs, but 
also manually handle the memory (de)allocations and transfers between the host 
program and the GPU device.  All these difficulties in the development of GPU 
programs, compared to normal (sequential) CPU programming, severely hinders 
the adoption of GPU programming for the masses.

%Even though most programming languages support GPU programming through various
%libraries, there are several solutions that offers GPU programming through high
%level programming -- the users still have to write their own kernels in some
%form, and likewise declare their own buffers.
%
For example, \csharp{} and \fsharp{} are two mainstream languages which, to
our knowledge, lack high-level (easy-to-use) solutions for GPU programming.
%
It is safe to say that there exists plenty of large, real-world projects
developed in \csharp{} and \fsharp{}, which could greatly benefit from 
accelerating their computational kernels on GPUs, but current, low-level
solutions (such as CUDA and OpenCL) would demand that those parts are 
rewritten as GPU code, and linked through foreign interfaces. 
This process affects code modularity and maintainability, and also requires
expert-GPU programmers; often enough this associated cost is deemed too high,
so the acceleration benefits are discarded in favor of maintaining a more 
accessible code base.

{\bf Relation to Related Work.}
%
A rich body of scientific work has been dedicated to the design of high-level 
programming models aimed primarily at GPU acceleration.  
%
For example, many domain-specific language (DSL) solutions target specific 
applications from areas such as image processing~\cite{Halide}, data 
analytics~\cite{HPAT}, deep-learning~\cite{DeliteDSLs}, mesh 
computations~\cite{OP2-Mesh}, iterative stencils~\cite{tang2011pochoir}. 

Various embedded libraries (DSLs) have been developed to allow high-level
expression and transparent-GPU acceleration of (simple) computational 
kernels written in a host (mainstream) language. For example, 
Java $8$ supports a stream (flat-parallel) construct~\cite{StreamJava8},
and Accelerate~\cite{Accelerate-DAMP,AccelerateStreaming} and 
Obsidian~\cite{svensson2011obsidian} are embedded in Haskell and support 
flat parallelism by means of operators such as map, reduce, scan, filter.
%
Other data-parallel and hardware-independent languages, such as 
Nesl~\cite{blelloch1994implementation,Bergstrom:2012:NDG:2398856.2364563},
Futhark~\cite{pldi17,Futhark:redomap,Futhark-ICFP18}, 
SAC~\cite{SaCShared2005,GrelSchoIJPP06}, and Lift~\cite{Lift-CGO17,Lift-ICFP}
support more complex parallel patterns (e.g., nested parallelism), 
and follow a standalone design.
% Cosmin says: ToDo: mention and add citations to Forma and Ebb in the 
%                    paragraphs above. 

%Currently, there does exist plenty of high level solutions to this problem.
%In particular, numerous domain specific languages exists that allows programmers
%to solve their domain specific problems in a high level language, and compile it
%to standalone GPU accelerated libraries or programs.
%
%Of such DSLs we have for instance:
%\begin{itemize}
%\item Forma
%\item Ebb
%\item one more
%\end{itemize}

In essence, all reviewed solutions are primarily aimed at developing (often) 
complementary techniques for optimizing parallelism in the context of GPU 
execution, but they do not address well the practical problem of integrating 
the use of the (now) accelerated computational kernels across different 
mainstream, productivity-oriented (programming) environments.
At best, they provide integration within one language (the host), 
and at the worst they require the user to hack an interoperability 
solution (based on a rudimentary foreign-function interface of the host). 

%However, DSLs such as these are always either embedded in some host language
%(such as Accelerate), or compiled to standalone executable programs. Their
%compilers are designed to optimize performance, but rarely to support
%interoperability to any significant degree.

This thesis investigates the latter, less-studied problem: we assume that
there exists a standalone data-parallel language that can express and 
execute efficiently computational kernels on GPUs (i.e., Futhark), and 
we study two transpilation techniques aimed at aggressively integrating
these kernels within the \fsharp{} environment, ideally to the extent that
the user is not even aware that Futhark is even part of the picture.

In this we take inspiration from early compiler 
solutions~\cite{alma:ISSAC,mapal_synasc} for interoperating
across computer-algebra systems, in which for example
the large libraries of Aldor~\cite{aldor}---a strongly and statically typed 
language supporting dependent types---have been seamlessly made available 
for use from Maple~\cite{maple_guide}---a very dynamic language aimed at 
ease of inspection and scripting.
%
Finally, perhaps the work most related to ours
is the experiment~\cite{apltail} that showed that programs 
written in a usefully-large subset of the old and interpreted (hence slow) APL 
language~\cite{dyalogbook} can be automatically translated to Futhark, 
optimized for GPU execution, and easily integrated in a mainstream application
written in Python. This thesis aims at providing a similar solution in which
the automatically-translated and mainstream-target languages are \fsharp{}
and \csharp{}, respectively.

 
%Projects like APLtail\cite{apltail} have shown new ways to obtain GPU-accelerated
%executables and libraries from source code written in  high level language. APLtail parses
%and compiles APL\footnote{more accurately a subset of the APL language} code into redistributable C or Python libraries.

In summary, the hardware for massively parallel programming is widely available,
and many solutions exists for writing efficient GPU programs in high level
languages. The problem is that these solutions do not allow integration across
different mainstream, productivity-oriented languages.



\section{What \fshark{} sets out to do}

This thesis takes inspiration from the APL-Futhark-Python experiment~\cite{apltail}, 
and creates an interoperability solution between \fsharp{}, Futhark, and \csharp{}.
The ultimate goal is (1) to allow computational kernels to be written in a usefully-large
{\em subset} of the \fsharp{} language, which can be automatically (and efficiently)
translated to Futhark, and (2) to also allow the resulting code to be integrated
back into a mainstream \csharp{} or \fsharp{} application. 
The advantages of this approach are that:
\begin{itemize}
    \item it requires a gentle learning curve for the user, who needs not learn Futhark, 
    \item it allows full access to \fsharp{}'s productivity tools, e.g., debugger, IDE,
            which is very useful when developing the computational kernels, and 
    \item it allows the code to be optimized by an aggressive compiler (Futhark),
            specialized in efficiently mapping nested data-parallelism to GPU hardware.  
\end{itemize}

%This thesis takes inspiration from the APL-Futhark-Python experiment~\cite{apltail}, 
%and creates an interoperability solution that lets users compile efficient GPU 
%programs from a high level programming
%language, whilst at the same time supporting a high level of interoperability
%with a mainstream language.
%Whereas APLtail allows integration of GPU Futhark-written computation kernels in
%C- and Python programs (by means of code generation), we would like to use code generation to make Futhark-written kernels available for use in \csharp{} and \fsharp{} programs.

The solution proposed in this thesis is organized in two steps. 
{\em The first step} is relatively straightforward and it refers to designing a \csharp{} 
code generator (backend) for the Futhark compiler. Starting from a Futhark specification,
the code generator must be able to generate \csharp{} source
files, that can be compiled and used either as standalone executables, or as
importable (\csharp{}) libraries in any other \csharp{} or \fsharp{} program.
There were several notable challenges in this process, namely 
\begin{itemize}
    \item designing a layout that can encapsulate the exports of a Futhark program
            in a single \csharp{} class, 
    \item designing static and efficient helper libraries that help and simplify
            the code-generation process, and
    \item designing a way to write sequential (non-GPU) Futhark code as pure 
            \csharp{} code, in cases where GPU devices are unavailable.
\end{itemize}

{\em The second step} is more challenging and refers to identifying a useful 
subset of \fsharp{} that can be automatically transpiled to Futhark, and, of
course, designing and implementing the transpilation scheme.
%
%The code generator alone should not convince anyone that we are creating GPU
%kernels from a high level language, which is why we also design and implement a
%compiler which takes source code written in a mainstream language, compiles it
%as efficient GPU kernels, and (together with the code generator) makes the
%resulting GPU program immediately operable from the mainstream language itself.  
The main challenges here were:
\begin{itemize} 
    \item identifying which parts of the \fsharp{} language can be suitably
             translated to Futhark and what the translation should be, 
    \item implementing a standard library written in \fsharp{}, which contains
            the implementation of Futhark's data-parallel operators.\footnote{
             For example \fsharp{} does not support a {\tt reduce} operator of type 
            $(\alpha\rightarrow\alpha\rightarrow\alpha)~\rightarrow~\alpha~\rightarrow~[\alpha]~\rightarrow~\alpha$.}
            On the one hand this library publishes/fixes the types of the
            data-parallel operators, and, on the other hand, it provides an 
            implementation semantically-equivalent to the one of Futhark, 
            for the case when the user is prototyping the application
            solely within the \fsharp{} environment.
    \item designing and implementing a compiler pipeline that would allow
            efficient and transparent GPU execution of \fsharp{} programs
            (via transpilation to Futhark), without the need to manually 
            call Futhark compilers or to import external libraries.
\end{itemize}

Finally, we demonstrate the feasibility of the approach by means of empirical validation:
We show that unit tests written in the identified subset of \fsharp{} can be compiled and 
executed correctly as computational kernels on the GPU. More convincingly, we also take 
several complex benchmark programs written in Futhark, we translate them to the identified
subset of \fsharp{}, and show that the GPU-compiled program, not only runs correctly,
but also achieves performances comparable to the one written natively in Futhark.
Finally, we show that the exports of the computational kernels originally written in 
either Futhark or the translatable subset of \fsharp{}, can be easily integrated back
in an application written in \csharp{}.

\clearpage

\section*{The contributions of this thesis}
The contributions of this thesis are as follows:
\begin{enumerate}
\item A \csharp{} code generator for the Futhark language compiler, which
  generates GPU accelerated libraries that can integrate seamlessly in 
  \csharp{} and \fsharp{} code bases.

\item A select subset of the \fsharp{} language which can be translated directly to
  Futhark source code of equivalent functionality. This includes a
  library which implements Futhark SOACs~\cite{pldi17} in \fsharp{}, allowing
  people to write \fsharp{} code which can be ported automatically to Futhark.

\item A compiler and wrapper pipeline which allows users to compile individual
  \fsharp{} modules in their projects to GPU accelerated libraries, and load and
  execute code from these modules in the rest of the \fsharp{} project.

\item A set of benchmarks and unit tests that shows that this approach is indeed feasible.
\end{enumerate}

\clearpage
\section*{Vocabulary}
Unless otherwise specified, these are the terms used in the thesis:
\begin{description}
\item[For \fshark{}]\hfill
  \begin{itemize}
  \item The \fshark{} \textit{subset} is the subset of the \fsharp{} language
    that is supported by the \fshark{} compiler.

  \item The FShark Prelude is the library of \fsharp{}-ported Futhark array
    functions and SOACs, and is included with \fshark{}.

  \item \fshark{} code is \fsharp{} code which exclusively uses the \fshark{}
    subset and FSharkPrelude.

  \item \fshark{} modules are \fsharp{} modules written entirely in \fshark{}
    code.
    
  \item \fshark{} projects are \fsharp{} projects which uses \fshark{} and
    \fshark{} modules.
    \\
  \end{itemize}

\item[For Futhark]\hfill
  \begin{itemize}

  \item Futhark code is code written in Futhark.

  \item Futhark C-, Python- or \csharp{} code refers to Futhark code that has been compiled
    into C-, Python or \csharp{} source code.

  \end{itemize}
\end{description}



\section*{Roadmap}
The main part of this thesis is split in four parts.
blaaaah


%%% Local Variables:
%%% mode: latex
%%% TeX-master: "../thesis"
%%% End:
