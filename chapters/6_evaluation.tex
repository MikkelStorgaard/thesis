\chapter{Benchmarks and evaluation}
\section*{\fshark{} generated Futhark compared to original Futhark code}

Appendices show 
\section*{The \texttt{LocVolCalib} benchmark}
small.in:
FShark (openCL) took 211882 microseconds.
Average invokation (fshark non openCL) time was 81194767 microseconds.
Native took 438929311 microseconds.

medium.in:
(Fshark opencl)invokation time was 310833 microseconds
Fshark nonopencl Average invokation time was 154141321 ms
Native took 900643005 microseconds.

large.in:

fshark sans opencl 2450637053 microseconds
Native took 24757874577 microseconds.


for all three datasets


\section*{The \texttt{nbody} benchmark}

for all three datasets


\subsection*{Specifications for benchmark}
We have run the benchmarks on a system with these attributes:
\begin{itemize}
\item CPU: 4 cores of Intel Core i5-6500 at 3.20GHz
  \begin{itemize}
  \item L1 cache: 128 KiB 
  \item L2 cache: 1024 KiB 
  \item L3 cache: 6144 KiB 
  \end{itemize}
\item GPU: GeForce GTX 970
\end{itemize}


Introduction for the two benchmarks LocVolCalib and nbody



why are they faster in general



%%% Local Variables:
%%% mode: latex
%%% TeX-master: "../thesis"
%%% End: