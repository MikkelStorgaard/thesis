\chapter{Array handling in FShark}
% introduction
\fsharp{} is a functional programming language on top of the .NET framework, which
means that it's primitive types like \texttt{int, float} and \texttt{list} all
correspond to already existing classes in the .NET framework. For example,
\fsharp{}'s \texttt{int} is an alias for .NET's \texttt{System.Int32} and
\texttt{float} is an alias for \texttt{System.Double}.

Therefore, we also find corresponding .NET classes for both \fsharp{}
\texttt{list}s and \texttt{array}s. \texttt{list}s are
\texttt{FSharp.Collections.FSharpList}s, and \texttt{array}s are
\texttt{System.Array}. (Note that \texttt{FSharpList}s is available from any
.NET framework language, as long as the corresponding assembly is referenced).

Although it is common to use lists in functional programs, the \fsharp{} subset
covered by FShark does not include lists -- In Futhark, and therefore also
FShark, our main goal is not handling list elements one after another, but
rather parallelizing computations across entire arrays of data simultaneously.

The \texttt{FSharp.Collections.FSharpList} is implemented as a singly-linked
list. SOACs called on singly-linked lists are inherently unparallelizable, as
the SOACs must traverse the list sequentially.
For example, calling $\texttt{map} f$ on a singly linked list \texttt{(x::xs)}
means computing $f \texttt{x}$ and inserting the result into $\texttt{map} f
\texttt{xs}$ recursively. We can do some parallel computations for these kinds
of SOACs, i.e. by making the main thread traverse the list and spawn
a thread for each element computation. However we will still have suboptimal
memory access performance, as the elements in the singly linked list doesn't
have any guarantees regarding their location in RAM, which means we are going to
perform many more memory loads compared to if we were performing calculations on
elements in a sequentially stored array elements in RAM.


% a basic description of lists in F# and arrays in F#

% short description of multidims in F#
% description of jagged arrays in F#
% description of array handling in Futhark 

% a description of SOACs in FSharkPrelude, analysis of various time complexities
% for FSharkPrelude SOACs

%% handling irregular arrays at runtime when using FSharkPrelude.

% Wrapping and unwrapping when invoking functions:
%% Why do we need to completely flatten arrays before passing into Futhark
%% module?
%%% Array flattening algorithm in pseudocode
%%% Cost of algorithm


% Reshaping flat array into jagged array
%% Description of algorithm
%% Cost of algorithm

% performance??

% alternative solutions: FSharkArrays

%%% Local Variables:
%%% mode: latex
%%% TeX-master: "../thesis"
%%% End: