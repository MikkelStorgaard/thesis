\chapter{The \fshark{} language}
\label{chap:fsharklanguage}

% introduction

\begin{figure}
  \centering
  \begin{tabular}{lclr}
    $t$ & $::=$ & $\texttt{int8}~|~\texttt{int16} ~|~ \texttt{int} ~ |~\texttt{int64} $ & (Integers) \\
        & $|$   & $\texttt{uint8} ~ | ~\texttt{uint16} ~|~\texttt{uint} ~|~\texttt{uint64} $ & (Unsigned integers) \\
        & $|$   & $\texttt{single} ~| ~\texttt{double}$ & (Floats) \\
        & $|$   & $\texttt{bool}$ & (Booleans) \\
        & $|$   & $(t_0 * \ldots * t_n)$ & (Tuples) \\
        & $|$   & $\{\texttt{id}_0:t_0 ; \ldots ; \texttt{id}_n:t_n\}$ & (Records) \\
        & $|$   & $t_0$ \texttt{array} & (Arrays) \\
    \\

    $k$ & $::=$ & $n$ & (Integer) \\
        & $|$   & $f$ & (Float) \\
        & $|$   & $b$ & (Boolean) \\
        & $|$   & $(k_0 , \ldots , k_n)$ & (Tuple) \\
        & $|$   & $\{\texttt{id}_0=k_0 ; \ldots ; \texttt{id}_n=k_n\}$ & (Record) \\
        & $|$   & $[k_0 ; \ldots ; k_n]$ & (Array) \\
    \\

    $p$ & $::=$ & $\texttt{id}$ & (Name pattern) \\
        & $|$   & $(p_0, \ldots, p_n)$ & (Tuple pattern) \\
  \end{tabular}
  \caption{The FShark syntax}
\end{figure}

\begin{figure}
  \centering
  \begin{tabular}{lclr}
    $e$ & $::=$ & $(e_0)$ & \\
        & $|$   & $k$ & Constant \\
        & $|$   & $v$ & Variable \\
        & $|$   & $(k_0 , \ldots , k_n)$ & (Tuple expression) \\
        & $|$   & $\{\texttt{id}_0=k_0 ; \ldots ; \texttt{id}_n=k_n\}$ & (Record expression) \\
        & $|$   & $[k_0 ; \ldots ; k_n]$ & (Array expression) \\
        & $|$   & $e_1 \odot e_2$ & (Binary operator) \\
        & $|$   & $-e$ & (Prefix minus) \\
        & $|$   & \texttt{not} $e$ & (Logical negation) \\
        & $|$   & \texttt{if} $e_1$ \texttt{then} $e_2$ \texttt{else} $e_3$ & (Branching) \\
        & $|$   & $v.[e_0] \ldots .[e_n]$ & (Array indexing) \\
        & $|$   & $v$.\texttt{id} & (Record indexing) \\
        & $|$   & $v_0.v_1$ & (Module indexing) \\
        & $|$   & \texttt{let} $p = e_1$ \texttt{in} $e_2$ & (Pattern binding) \\
        & $|$   & $v$ $e_0$ $\ldots$ $e_n$ & (Function call) \\
    \\
    $fun$ & $::=$ & \texttt{[<FSharkEntry>]} $\texttt{let}\ v\ (v_1 : t_1)\ \ldots\ (v_n : t_n) : t = e$ & \\
        & $|$   & $\texttt{let}\ v\ (v_1 : t_1)\ \ldots\ (v_n : t_n) : t' = e,$ & \\
        &       & \hspace{1em} \textit{(for any $i \in {1..n}$, $t_i$ is not a tuple)} \\
    \\

    $typealias$ & $::=$ & $\texttt{type}\ v\ = t $& \\
    $module$ & $::=$ & $\texttt{module}\ v = prog'\ progs'$ & \\
    \\
    $prog$ & $::=$ & $module\ prog$ & \\
           & $|$   & $prog'\ prog$  & \\
           & $|$   & $\epsilon$     & \\

    $prog'$ & $::=$ & $typealias$   & \\
            & $|$   & $fun$ & \\

    $progs'$ & $::=$ & $prog'\ progs'$   & \\
             & $|$   & $\epsilon$ & \\
  \end{tabular}
  \caption{The FShark syntax, expressions}
\end{figure}

\subsection{\fsharp{} operators available in \fshark{}}
The \fsharp{} subset chosen for \fshark{} is described in this subsection.
\begin{description}
\item[Arithmetic operators]\hfill\\
  The set of supported arithmetic operators is addition (\texttt{+}),
  binary subtraction and unary negation (\texttt{-}), multiplication
  (\texttt{*}), division (\texttt{/}) and modulus (\texttt{\%}).
  
\item[Boolean operators]\hfill\\
  \fshark{} currently supports logical AND (\texttt{\&\&}), logical OR
  (\texttt{\|\|}), less- and greater-than (\texttt{<}, \texttt{>}), less- and
  greater-or-equal (\texttt{<=}, \texttt{>=}), equality (\texttt{=}),
  inequality (\texttt{<>}) and logical negation (\texttt{not}).

\item[Special operators]\hfill\\
  \fshark{} also supports some of \fsharp{}s syntactic sugar. These operators
  might not have direct Futhark counterparts, but their applications can be
  rewritten in Futhark for equivalent functionality.
  The supported operators are back- and forward pipes (\texttt{<|} and
  \texttt{|>}), and the range operator ($e_0$ \texttt{..} $e_1$), which
  generates the sequence of numbers in the interval $[e_0,e_1]$. Note that in
  \fshark{}, the range operator must be used inside an array as so
  \texttt{[|$e_0$..$e_1$|]} so we adhere to using arrays and not lists in our
  \fshark{} programs.
\end{description}
Note that all of these operators are overloaded and defined for all integer
and floating point types in \fsharp{}.



\subsection{\fsharp{} standard library functions available in \fshark{}}
\fshark{} supports a subset of the \fsharp{} standard library. These are
functions that are imported in \fsharp{} modules by default.

\begin{description}
\item[\texttt{id}]\hfill\\
  The identity function.

\item[Common math function]\hfill\\
  The square root function (\texttt{sqrt}), the absolute value (\texttt{abs}),
  the natural exponential function (\texttt{exp}), the binary- and the decimal
  logarithm (\texttt{log} and \texttt{log10}).
  
\item[Common trigonometric functions]\hfill\\
  Sine, cosine and tangent functions (both standard and hyperbolic):
  \texttt{sin}, \texttt{cos}, \texttt{tan}), \texttt{sinh}, \texttt{cosh} and \texttt{tanh}.
  Also one- and two-argument arctangent: \texttt{atan} and \texttt{atan2}.

\item[Rounding functions]\hfill\\
  \fshark{} supports all of \fsharp{}s rounding functions:
  \texttt{floor}, \texttt{ceil}, \texttt{round} and \texttt{truncate}.
  
\item[Various common number functions]\hfill\\
  \texttt{min}, \texttt{max}, \texttt{sign} and \texttt{compare}.

  

\end{description}



% why these
%% we can express all things, the mathematical standard funs have direct
%% equivalents in Futhark
% this should be extended with bitwise operators
% some arithmetic functions have been implemented using inline identities




\begin{figure}
  \centering
  \begin{tabular}{lclr}
    $e$ & $::=$ & $k$ & Constant \\
        & $|$   & $v$ & Variable \\
        & $|$   & $(k_0 , \ldots , k_n)$ & (Tuple expression) \\
        & $|$   & $\{\texttt{id}_0=k_0 ; \ldots ; \texttt{id}_n=k_n\}$ & (Record expression) \\
        & $|$   & $[k_0 ; \ldots ; k_n]$ & (Array expression) \\
        & $|$   & $e_1 \odot e_2$ & (Binary operator) \\
        & $|$   & $-e$ & (Prefix minus) \\
        & $|$   & \texttt{not} $e$ & (Logical negation) \\
  \end{tabular}
  \caption{f binary operators}
\end{figure}

\begin{figure}
  \centering
  \begin{tabular}{lclr}
    $e$ & $::=$ & $k$ & Constant \\
        & $|$   & $v$ & Variable \\
        & $|$   & $(k_0 , \ldots , k_n)$ & (Tuple expression) \\
        & $|$   & $\{\texttt{id}_0=k_0 ; \ldots ; \texttt{id}_n=k_n\}$ & (Record expression) \\
        & $|$   & $[k_0 ; \ldots ; k_n]$ & (Array expression) \\
        & $|$   & $e_1 \odot e_2$ & (Binary operator) \\
        & $|$   & $-e$ & (Prefix minus) \\
        & $|$   & \texttt{not} $e$ & (Logical negation) \\
  \end{tabular}
  \caption{FShark SOACs}
\end{figure}

LANGUAGE REFERNCE

%%% Local Variables:
%%% mode: latex
%%% TeX-master: "../thesis"
%%% End: