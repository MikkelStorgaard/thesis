\chapter{Current limitations}
In the chapter, we describe the current known limitations of both our code
generator, the \fshark{} language design and of the \fshark{} compiler. The
limitations are divided into two categories; those caused by our design choices,
and those caused by a lacking implementation.

\section{The \csharp{} code generator}
Both of the code generator limitations listed below are caused by lack of
implementation.

\subsection{Cumbersome array entry functions in Futhark libraries}
\label{cumbersomearrays}
As described in sec. \ref{subsec:flatarraysinentryfuncs}, we currently have to
flatten our jagged arrays before we can pass them to our Futhark library
functions. Likewise, we have to unflatten the results if we want to use them as
jagged arrays again afterwards.

In sec. \ref{sec:convertingarrays}, we presented a solution for both flattening
and unflattening such arrays, and thus solving this limitation is merely a
question of porting and implementing these algorithms in the Futhark generated
\csharp{} libraries.

\subsection{Unnecessary memory allocations in chained Futhark function calls}
The current implementation of the code generator causes significant overhead
when chaining together GPU function calls, as discussed in sec.
\ref{invocationoverhead}.
Whilst not being a functional limitation, implementing an
opaque return type for Futhark GPU calls would increase runtime
performance in any programs that chain together such calls.

The Python code generator for Futhark already has such an opaque data type
implemented, and one could look to this implementation for inspiration on how to
design a similar data type for Futhark \csharp{}.

\section{The \fshark{} language}


\section{The \fshark{} compiler}

\subsection{Disallowing certain types of \fshark{} entry functions}
As the \fshark{} wrapper relies on the flattening algorithms shown in sec.
\ref{sec:convertingarrays} to make \fsharp{}s jagged arrays compatible with
Futhark's flat arrays (sec. \ref{subsec:flatarraysinentryfuncs}), we currently
prohibit \fshark{} entry functions have return types that are either
\texttt{('a[] * int64[])} tuples, or tuples or arrays that contains such
tuples. This is described in detail in subsec. \ref{subsec:hinderedtupletype}.

This could be solved by moving the array flattening into the generated Futhark
\csharp{} libraries as described in subsec. \ref{cumbersomearrays}, solving two
limitations at the same time.


%%% Local Variables:
%%% mode: latex
%%% TeX-master: "../thesis"
%%% End:
