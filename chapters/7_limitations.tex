\chapter{Current limitations}
\label{sec:currentlimitations}
In the chapter, we describe the current known limitations of both our code
generator, the \fshark{} language design and of the \fshark{} compiler. The
limitations are divided into two categories; those caused by our design choices,
and those caused by a lacking implementation.

\section{The \csharp{} code generator}
Both of the code generator limitations listed below are caused by lack of
implementation.

\subsection{Errors in the implementation}
\label{sec:errorsintheimplementation}
Using Futhark's own test suite, we have tested and evaluated the current
implementation of the code generator.
These tests have made it apparent that the \csharp{} code
generator does not yet generate fully correct Futhark programs. 

The Futhark test suite contains 961 tests, which tests everything from the
Futhark compiler itself (for example whether type aliases,
higher order modules et al. handled correctly?),
to whether the individual mathematical and bitwise operations are correctly
translated from Futhark to desired target language (\csharp{} in our case), and
whether types, be they array types or scalar types, are retained throughout the
entire program.
\\
It also tests the \texttt{stdin/stdout} functionality of the generated programs.
\\
The current implementation does not pass all tests correctly: for the Futhark
\csharp{} compiler for non-OpenCL programs, we are passing 840 out of 961 tests.\\
For the Futhark \csharp{} compiler for OpenCL programs, we are passing just 833
out of 961 tests.

For many of these tests, we can immediately see what the issue is. In example,
we our stdin reader for \csharp{} programs does not currently read empty
arrays\footnote{such as example \texttt{empty(i32)} or \texttt{empty(f32)}.}
correctly. Such errors are caused by lack of implementation in the stdin reader, and can be
corrected with relative ease by implementing the missing features.\\
The test suite also reveals various \textit{off-by-one} errors in edge cases of
the implementation, which should be simple to debug and fix.

Despite of these temporary bad test results, we are still confident that our
code generator is mostly correct. After all, the current implementation does
pass more than 86\% of the test suite.
\\
More importantly, although we have 128 failing tests, the test results also show
that many of these tests have the same point of failure.

For example, we have 16 failing tests caused by the missing reader features, and 8
\texttt{off-by-one} errors stemming from a certain helper function in the
\csharp{} class.
\\
If this is representative for the other failing tests as well, it is likely that
the remaining number of (known) bugs in the implementation is closer to ten, than
to a hundred.

\subsection{Errors in the benchmarking functionality}
\label{sec:errorsinbenchmark}
The current implementation contains an error which hinders us in reliably
testing all benchmarks in the Futhark benchmark suite. In rare cases, a
benchmark program will both compile and execute correctly, but
return the wrong runtime measurements.

In principle, this bug should disqualify us from being able to report any
reliable benchmarks from our solution at all, but repeated testing shows that
the bug is indeed only present for certain benchmark programs.

For example, the \texttt{Crystal} benchmark executes correctly for both Futhark
\clang{} and Futhark \csharp{}, but the runtimes reported are obviously
wrong for the Futhark \csharp{} case:

\begin{minted}[breaklines]{bash}
$ futhark-bench --compiler=futhark-opencl crystal.fut
Compiling ./crystal.fut...
Results for ./crystal.fut:
dataset #0 ("200i32 30.0f32 5i32 1i32 1.0f32"):        32.00us (avg. of 10 runs; RSD: 0.37)
dataset #1 ("20i32 30.0f32 5i32 50i32 0.5f32"):        21.20us (avg. of 10 runs; RSD: 0.04)
dataset #2 ("40i32 30.0f32 5i32 50i32 0.5f32"):        40.70us (avg. of 10 runs; RSD: 0.02)
dataset #3 ("40i32 30.0f32 50i32 50i32 0.5f32"):      263.40us (avg. of 10 runs; RSD: 0.27)
dataset #4 ("2000i32 30.0f32 50i32 1i32 1.0f32"):   10800.40us (avg. of 10 runs; RSD: 0.04)

$ futhark-bench --compiler=futhark-csopencl crystal.fut
Compiling ./crystal.fut...
Results for ./crystal.fut:
dataset #0 ("200i32 30.0f32 5i32 1i32 1.0f32"):        32.20us (avg. of 10 runs; RSD: 0.54)
dataset #1 ("20i32 30.0f32 5i32 50i32 0.5f32"):        31.90us (avg. of 10 runs; RSD: 0.58)
dataset #2 ("40i32 30.0f32 5i32 50i32 0.5f32"):        29.30us (avg. of 10 runs; RSD: 0.57)
dataset #3 ("40i32 30.0f32 50i32 50i32 0.5f32"):       31.60us (avg. of 10 runs; RSD: 0.65)
dataset #4 ("2000i32 30.0f32 50i32 1i32 1.0f32"):      28.70us (avg. of 10 runs; RSD: 0.62)
\end{minted}

On the other hand, the Hotspot benchmark works as expected:
\begin{minted}{fsharp}
$ futhark-bench --compiler=futhark-opencl hotspot.fut  
Compiling ./hotspot.fut...
Results for ./hotspot.fut:
dataset data/64.in:      1686.00us (avg. of 10 runs; RSD: 0.07)
dataset data/512.in:    13064.60us (avg. of 10 runs; RSD: 0.01)
dataset data/1024.in:   42276.30us (avg. of 10 runs; RSD: 0.01)
$ futhark-bench --compiler=futhark-csopencl hotspot.fut
Compiling ./hotspot.fut...
Results for ./hotspot.fut:
dataset data/64.in:      2172.10us (avg. of 10 runs; RSD: 0.06)
dataset data/512.in:     9172.20us (avg. of 10 runs; RSD: 0.42)
dataset data/1024.in:   33946.10us (avg. of 10 runs; RSD: 0.46)
\end{minted}



\subsection{Cumbersome array entry functions in Futhark libraries}
\label{cumbersomearrays}
As described in section~\ref{subsec:flatarraysinentryfuncs}, we currently have to
flatten our jagged arrays before we can pass them to our Futhark library
functions. Likewise, we have to unflatten the results if we want to use them as
jagged arrays again afterwards.

In sec. \ref{sec:convertingarrays}, we presented a solution for both flattening
and unflattening such arrays, and thus solving this limitation is merely a
question of porting and implementing these algorithms in the Futhark generated
\csharp{} libraries.

\subsection{Unnecessary memory allocations in chained Futhark function calls}
The current implementation of the code generator causes significant overhead
when chaining together GPU function calls, as discussed in sec.
\ref{invocationoverhead}.
Whilst not being a functional limitation, implementing an
opaque return type for Futhark GPU calls would increase runtime
performance in any programs that chain together such calls.

The Python code generator for Futhark already has such an opaque data type
implemented, and one could look to this implementation for inspiration on how to
design a similar data type for Futhark \csharp{}.

\section{The \fshark{} language}
\subsection{Scatter}
\label{subsec:badscatter}
The current implementation of the \fshark{} language allows users to use the
SOAC \texttt{Scatter}. However, the actual \texttt{scatter} SOAC from Futhark
has certain usage constraints concerning uniqueness\footnote{See \url{https://futhark.readthedocs.io/en/latest/language-reference.html\#in-place-updates}}, which we currently do not enforce
in \fshark{}.

This means that users can unknowingly write valid \fshark{} programs, that are
not valid Futhark programs, if they accidentally break the rules of scatter
usage. This is not a nice position for the \fshark{} language to be in, so we
should in principle either remove the \texttt{Scatter} SOAC from the \fshark{} standard
library, or enforce the uniqueness constraint in the \fshark{} compiler.
This might take some time.

\section{The \fshark{} compiler}

\subsection{Disallowing certain types of \fshark{} entry functions}
As the \fshark{} wrapper relies on the flattening algorithms shown in sec.
\ref{sec:convertingarrays} to make \fsharp{}s jagged arrays compatible with
Futhark's flat arrays (sec. \ref{subsec:flatarraysinentryfuncs}), we currently
prohibit \fshark{} entry functions have return types that are either
\texttt{('a[] * int64[])} tuples, or tuples or arrays that contains such
tuples. This is described in detail in subsec. \ref{subsec:hinderedtupletype}.

This could be solved by moving the array flattening into the generated Futhark
\csharp{} libraries as described in subsec. \ref{cumbersomearrays}, solving two
limitations at the same time.

\subsection{Allow compiler usage outside of \fshark{} wrapper}
As discussed in section~\ref{invocationoverhead}, we would like to be able to use just the \fshark{}
compiler, without having to go through the \fshark{} wrapper.
Our goal is to be able to compile \fshark{} programs directly from the
command-line like so:
\begin{minted}{bash}
$ fshark -o MyModule.dll MyModule.fs
\end{minted}
This should be relatively easy to achieve, as the compiler architecture already
exists within the \fshark{} project.

\section{The \fshark{} validation}
The current \fshark{} test suite lacks accompanying tests for the \fshark{} type
conversion functions. These must be implemented to assure us that \fshark{}
correctly translate type convertion functions to Futhark.


%%% Local Variables:
%%% mode: latex
%%% TeX-master: "../thesis"
%%% End:
