\documentclass[10pt, a4paper]{report}
\usepackage[utf8]{inputenc}
\usepackage{graphicx}
\usepackage{blindtext}
\usepackage{seqsplit}
\usepackage[cache=false]{minted}
\usepackage{listings}
\usepackage[htt]{hyphenat}
\usepackage{amsmath,amssymb,amsbsy}
\usepackage{graphicx}
\usepackage{amsmath,amssymb,amsbsy}
\usepackage{stmaryrd}
\usepackage{semantic}
\usepackage{url}
\usepackage{color}
\usepackage{flushend}
\usepackage{subcaption}
\usepackage{tikz}


\title{
	{FShark - something something subtitle}\\
	{\large University of Copenhagen}\\
}
\author{Mikkel Storgaard Knudsen}

\newcommand{\fshark}{\texttt{FShark}}
\newcommand{\fsharp}{\texttt{F\#}}
\newcommand{\csharp}{\texttt{C\#}}
\newcommand{\fsharpexpr}{\texttt{FSharpExpr}}
% Futhark syntax highlighting setup
\usepackage{listings}
\renewcommand{\ttdefault}{pcr} % Courier instead of Computer Modern
% Fix dashes in listings (from
% https://tex.stackexchange.com/questions/33185/listings-package-changes-hyphens-to-minus-signs
% )
\makeatletter
\lst@CCPutMacro\lst@ProcessOther {"2D}{\lst@ttfamily{-{}}{-{}}}
\@empty\z@\@empty
\makeatother
\lstdefinelanguage{Futhark}
{keywords={fun,if,then,else,loop,do,map,reduce,reduceComm,filter,scan,redomap,redomapComm,transpose,reshape,iota,replicate,let,in,for,while,with,f32,int,zip,streamSeq,zipWith,unsafe,streamRed,streamMap,mapPerThread,fn,reduceKernel,concat,split,size},%
  sensitive=true,
  comment=[l]{--},
  moredelim=**[is][\color{blue}]{$}{$}
}
\lstdefinelanguage{tail}
{keywords={let,in},%
  sensitive=true,
  comment=[l]{--},
  moredelim=**[is][\color{blue}]{$}{$}
}
\lstset{
  language=Futhark,
  basicstyle=\ttfamily,
  keywordstyle=\bfseries,
  showlines=true,
  columns=fullflexible,
  keepspaces=true,
}
% for adjustwidth environment
\usepackage[strict]{changepage}

% for formal definitions
\usepackage{framed}

% environment derived from framed.sty: see leftbar environment definition
\definecolor{formalshade}{rgb}{0.95,0.95,1}

\newenvironment{formal}{%
  \def\FrameCommand{%
    \hspace{1pt}%
    {\color{darkblue}\vrule width 2pt}%
    {\color{formalshade}\vrule width 4pt}%
    \colorbox{formalshade}%
  }%
  \MakeFramed{\advance\hsize-\width\FrameRestore}%
  \noindent\hspace{-4.55pt}% disable indenting first paragraph
  \begin{adjustwidth}{}{7pt}%
  \vspace{2pt}\vspace{2pt}%
}
{%
  \vspace{2pt}\end{adjustwidth}\endMakeFramed%
}

\newcommand{\evals}[1]{\llbracket #1 \rrbracket}
\newcommand{\evalfn}[1]{\evals{#1}_{\textrm{fn}}}
\newcommand{\evalbinop}[1]{\evals{#1}_{\textrm{binop}}}
\newcommand{\evalunop}[1]{\evals{#1}_{\textrm{unop}}}
\newcommand{\id}[1]{\emph{#1}}
\newcommand{\lit}[1]{\text{\ttfamily #1}} % literal

\definecolor{rosso}{RGB}{220,57,18}
\definecolor{giallo}{RGB}{255,153,0}
\definecolor{blu}{RGB}{102,140,217}
\definecolor{verde}{RGB}{16,150,24}
\definecolor{viola}{RGB}{153,0,153}

\makeatletter

\tikzstyle{chart}=[
    legend label/.style={font={\scriptsize},anchor=west,align=left},
    legend box/.style={rectangle, draw, minimum size=5pt},
    axis/.style={black,semithick,->},
    axis label/.style={anchor=east,font={\tiny}},
]

\tikzstyle{bar chart}=[
    chart,
    bar width/.code={
        \pgfmathparse{##1/2}
        \global\let\bar@w\pgfmathresult
    },
    bar/.style={very thick, draw=white},
    bar label/.style={font={\bf\small},anchor=north},
    bar value/.style={font={\footnotesize}},
    bar width=.75,
]

\tikzstyle{pie chart}=[
    chart,
    slice/.style={line cap=round, line join=round, very thick,draw=white},
    pie title/.style={font={\bf}},
    slice type/.style 2 args={
        ##1/.style={fill=##2},
        values of ##1/.style={}
    }
]

\pgfdeclarelayer{background}
\pgfdeclarelayer{foreground}
\pgfsetlayers{background,main,foreground}


\newcommand{\pie}[3][]{
    \begin{scope}[#1]
    \pgfmathsetmacro{\curA}{90}
    \pgfmathsetmacro{\r}{1}
    \def\c{(0,0)}
    \node[pie title] at (90:1.3) {#2};
    \foreach \v/\s in{#3}{
        \pgfmathsetmacro{\deltaA}{\v/100*360}
        \pgfmathsetmacro{\nextA}{\curA + \deltaA}
        \pgfmathsetmacro{\midA}{(\curA+\nextA)/2}

        \path[slice,\s] \c
            -- +(\curA:\r)
            arc (\curA:\nextA:\r)
            -- cycle;
        \pgfmathsetmacro{\d}{max((\deltaA * -(.5/50) + 1) , .5)}

        \begin{pgfonlayer}{foreground}
        \path \c -- node[pos=\d,pie values,values of \s]{$\v\%$} +(\midA:\r);
        \end{pgfonlayer}

        \global\let\curA\nextA
    }
    \end{scope}
}

\newcommand{\legend}[2][]{
    \begin{scope}[#1]
    \path
        \foreach \n/\s in {#2}
            {
                  ++(0,-10pt) node[\s,legend box] {} +(5pt,0) node[legend label] {\n}
            }
    ;
    \end{scope}
}

\begin{document}

%%%% CHAPTERS
% \chapter{Prelude}

%%% Local Variables:
%%% mode: latex
%%% TeX-master: "../thesis"
%%% End:
\chapter{Introduction}
Developers worldwide are, and have always been, on the lookout for increased
computing performance.
Until recently, the increased performance could easily be
achieved through advances within raw computing power, as CPU's had steadily been
doubling their number of on-chip transistors, in rough accordance to Moore's Law (citér
her).

As performance increases in single-CPU design has stalled due to the power
wall\cite{powerwall} (among other things), developers are turning to multi-core
processors instead. As the number of cores increases, so does the number of
active threads available for parallel data processing.

Modern mainstream GPUs can run tens of thousands of threads in
parallel. Modern mainstream CPUs, like the current Ryzen series by AMD, usually
support between 10 and 20 simultaneous threads.
This makes GPUs the optimal target for data-parallel programming.

GPU programming is complicated: GPU-targeting developers must not only
write the computational kernels for the GPUs, but also often manually handle the
memory allocations and -transfers between the main program and the GPU device.
Such difficulties in GPU development, compared to normal (sequential) CPU
development, severely hinders the adaption of GPU programming in general.

Even though most programming languages support GPU programming through various
libraries, there are very solutions that offers GPU programming through high
level programming \-- the users still have to write their own kernels in some
form, and likewise declare their own buffers.

Two mainstream languages which lack high level GPU programming solutions are
\csharp{} and \fsharp{}. 

It is safe to say that there exists plenty of \csharp{} and \fsharp{} projects
in the real world, which could greatly benefit from parallelizing parts of their
algorithms, but current solutions would then demand that those parts in
particular
should be rewritten at least partly as GPU code, depending on the libraries
used. Depending on someone to have non-mainstream GPU coding skills on a
conventional developer team is not feasible, so the benefits from parallelizing
are often left alone in favor of maintaining a more accessible code base.
\clearpage

Currently, there does exist plenty of high level solutions to this problem.
In particular, numerous domain specific languages exists that allows programmers
to solve their domain specific problems in a high level language, and compile it
to standalone GPU accelerated libraries or programs.

Of such DSLs we have for instance:
\begin{itemize}
\item Forma
\item Ebb
\item one more
\end{itemize}

However, DSLs such as these are always either embedded in some host language
(such as Accelerate), or compiled to standalone executable programs. Their
compilers are designed to optimize performance, but rarely to support
interoperability to any significant degree.

Projects like APLtail\cite{apltail} have shown new ways to obtain GPU-accelerated
executables and libraries from source code written in  high level language. APLtail parses
and compiles APL\footnote{more accurately a subset of the APL language} code into redistributable C or Python libraries.

In summary, the hardware for massively parallel programming is widely available.
Furthermore, solutions exists for writing efficient GPU programs in high level
languages, but these have weak interoperability support with mainstream
languages.

\section{What \fshark{} sets out to do}
This thesis takes inspiration from APLtail\cite{apltail}, and creates a solution
that lets users compile efficient GPU programs from a high level programming
language, whilst at the same time supporting a high level of interoperability
with a mainstream language.
Whereas APLtail allows integration of GPU Futhark-written computation kernels in
C- and Python programs (by means of code generation), we would like to use code generation to make Futhark-written kernels available for use in \csharp{} and \fsharp{} programs.

To show that this is feasible, we first design a \csharp{} code generator for the
Futhark compiler. This code generator must be able to generate \csharp{} source
files, that can be compiled and used either as standalone executables, or as
importable libraries in any other \csharp{} or \fsharp{} program.
There were several notable challenges in this process, namely 1) designing
\csharp{} programs that could encapsulate entire Futhark programs in a single
class, and 2) designing helper libraries to include in the generated code, and
3) designing a way to write sequential (non-GPU) Futhark code as pure \csharp{},
in cases where GPU devices are unavailable.

The code generator alone should not convince anyone that we are creating GPU
kernels from a high level language, which is why we also design and implement a
compiler which takes source code written in a mainstream language, compiles it
as efficient GPU kernels, and (together with the code generator) makes the
resulting GPU program immediately operable from the mainstream language itself.  
The main challenges for this was 1) identifying which parts of the \fsharp{}
language that were suitable for Futhark translation, 2) implementing a standard
library for Futhark targeted \fsharp{} programs, and 3) designing and implementing a compiler
pipeline that would let users program and use GPU kernels in \fsharp{}, without
manually using Futhark compilers or importing external libraries.

Empirical evaluation demonstrates that this approach is feasible.
We both show that that unit tests written in a high level language can be compiled and executed
correctly as computational kernels on the GPU, just as we also take complex
benchmark programs written in a mainstream language, compile them into
computational kernels for the GPU, and use them directly in the mainstream
language afterwards.







\section*{Motivation}
\fshark{} is intended to be a way of writing and utilizing Futhark, without
actually having to write or interact with the Futhark language and compiler
itself. Besides some tooling and an \fsharp{} SOAC library, it primarily consists of the \fshark{} compiler that compiles from
\fsharp{} source code to Futhark source code, and the Futhark \csharp{}
generator, which compiles Futhark programs as either standalone \csharp{}
programs or -libraries.

As much as most developers are happy to increase performance on big
computations, it is not always an option to incorporate an extra langauge into
an already existing programming language. At this moment, using Futhark in
either a \csharp{}- or \fsharp{} project is a contrived process that usually
requires spawning a subprocess with a \texttt{futhark-opencl} C program from inside one of the .NET
projects.

In order to use Futhark natively in .NET languages, it is therefore
necessary to write a backend for Futhark in a .NET language.
For \fshark{}, I have chosen to implement this backend in \csharp{}, as the Futhark intermediate
code \texttt{ImpCode}\footnote{which stands for Imperative Code} is trivial to
translate into imperative \csharp{} statements and expressions.
Also, there are \csharp{} libraries available which supply OpenCL bindings, which are
needed to implement the necessary OpenCL constructs from \texttt{ImpCode}.

It is my belief that exporting Futhark programs as .NET executables and
-libraries will lower the barrier to Futhark usage in .NET projects
significantly, hopefully increasing the all-round number of Futhark users, and
in the long term, increasing utilization of GPU programming and making it more
widely available.

However, one could do even more than just exporting Futhark to .NET, to increase
accessibility:

As tens of thousands of programmers worldwide (CHECK NUMBER JEEEEZ) are already
writing \fsharp{} programs, and that most of \fsharp{}s functional language features can be
directly translated into equivalent Futhark features, it became worthwhile to
investigate whether it was possible to design a way for users to both write and
utilize Futhark in \fsharp{} projects, without ever actually touching the
Futhark language or compiler themselves.
Instead, users can write their data parallel \fsharp{} modules in \fshark{}, and compile these
modules into Futhark libraries automatically.

In this case, it would be possible to get Futhark speeds in \fsharp{} programs,
without doing much more than installing the Futhark compiler locally, and adding
the required \fshark{} libraries to the \fsharp{} project.

It is my belief that being able to achieve Futhark performance in regular \fsharp{}
programs almost automatically, will make it significantly easier for people to
adapt to Futhark programming.

(SOME MORE MORE SOME MORE)

\clearpage

\section*{The contributions of this thesis}
The contributions of this thesis are as follows:
\begin{enumerate}
\item A \csharp{} code generator for the Futhark language compiler, which
  generates GPU accelerated libraries that can integrate seamlessly in 
  \csharp{} and \fsharp{} code bases.

\item A select subset of the \fsharp{} langauge which can be translated directly to
  Futhark source code of equivalent functionality. This includes a
  library which implements Futhark SOACs\cite{soacs} in \fsharp{}, allowing
  people to write \fsharp{} code which can be ported automatically to Futhark.

\item A compiler and wrapper pipeline which allows users to compile individual
  \fsharp{} modules in their projects to GPU accelerated libraries, and load and
  execute code from these modules in the rest of the \fsharp{} project.

\item A set of benchmarks and unit tests that shows that this approach is indeed feasible.
\end{enumerate}

\clearpage
\section*{Vocabulary}
Unless otherwise specified, these are the terms used in the thesis:
\begin{description}
\item[For \fshark{}]\hfill
  \begin{itemize}
  \item The \fshark{} \textit{subset} is the subset of the \fsharp{} language
    that is supported by the \fshark{} compiler.

  \item The FShark Prelude is the library of \fsharp{}-ported Futhark array
    functions and SOACs, and is included with \fshark{}.

  \item \fshark{} code is \fsharp{} code which exclusively uses the \fshark{}
    subset and FSharkPrelude.

  \item \fshark{} modules are \fsharp{} modules written entirely in \fshark{}
    code.
    
  \item \fshark{} projects are \fsharp{} projects which uses \fshark{} and
    \fshark{} modules.
    \\
  \end{itemize}

\item[For Futhark]\hfill
  \begin{itemize}

  \item Futhark code is code written in Futhark.

  \item Futhark C-, Python- or \csharp{} code refers to Futhark code that has been compiled
    into C-, Python or \csharp{} source code.

  \end{itemize}
\end{description}



\section*{Roadmap}
The main part of this thesis is split in four parts.
blaaaah


%%% Local Variables:
%%% mode: latex
%%% TeX-master: "../thesis"
%%% End:
\chapter{Background}
\fshark{} is built on the interaction between Futhark, \fsharp{} and \csharp{},
wherefore WE SHOULD HAVE A BETTER INTRODUCTION FOR THIS CHAPTER.

\section{\fsharp{}}
\fsharp{} is a relatively young .NET-based language, first released in 2003.
It is a strongly-typed multiple-paradigm language, with a syntax that is
primarily functional, resembling OCaml.
Although \fsharp{} is not as widely used as \csharp{}, \texttt{Java} and the
like, it is currently experiencing increasing adaptation among
developers\cite{citeme}.
Besides supporting multiple paradigms and a reasonable subset of functional
langauge features (such as pattern matching), \fsharp{}s primary strength is
it's interoperability with the rest of the .NET ecosystem. Like \csharp{},
\fsharp{} programs are compiled into Microsoft's \texttt{Common Intermediate
  Langauge}, and executed using Microsoft's \texttt{Common Language Runtime}.

Therefore, \fsharp{} programs have full access to the standard .NET library,
just as it can also readily import and use classes and methods from arbitrary
\csharp{} libraries.

For \fshark{}, \fsharp{} has been selected as a source language for several
reasons.
First, most of \fsharp{}s syntax is readily translatable into Futhark syntax, as
long as the programmer stays away from using any of \fsharp{}s non-functional
constructs, like \texttt{async} or it's object oriented features.
Second, as \fsharp{} effortlessly interoperates with \csharp{} programs, and
\csharp{} has plenty of OpenCL libraries available, we can write imperative
OpenCL-powered programs in \csharp{}, for use in \fsharp{} projects.
%% written example


\section{Futhark}
Quoting from Futhark's own homepage,
\begin{formal}
  Futhark is a small programming language designed to be compiled to efficient parallel code. It is a statically typed, data-parallel, and purely functional array language in the ML family, and comes with a heavily optimising ahead-of-time compiler that presently generates GPU code via OpenCL, although the language itself is hardware-agnostic.
\end{formal}
So far, plenty of handwritten GPU benchmark programs implemented in CUDA et al,
has been ported to Futhark, with significant performance gains as a result.
\cite{citesomething}. With these results in mind, it makes sense to start
implementing other parallelizable algorithms and programs in Futhark. However,
in the grand scheme of things, Futhark is still a relatively obscure programming
language, and is almost solely used in academic settings.

With Futhark being a purely functional programming language, it has very few
imperative language constructs available, and the few that it has, like
in-place updates, are merely syntactic sugar for other existing library function calls.

As Futhark's main functionality is generating OpenCL kernels, it is in principle
possible to compile Futhark programs for any language that are able to interface
with the OpenCL API.

%err, måske flytte eller skrive om
As a target language for \fsharp{} translations,
Futhark is ideal as we can identify and relatively easily translate a subset
of the \fsharp{} language to equivalent Futhark code, as the syntax itself is
very similar. Even though \fsharp{} also allows plenty of imperative and object
oriented programming,
\fshark{} blocks the user from using these constructs, by failing at \fshark{}
compile time.

% måske tilføje noget om unsafe

\section{\csharp{}}
\csharp{} 
\csharp{} 
\csharp{} 
\csharp{} 




%%% Local Variables:
%%% mode: latex
%%% TeX-master: "../thesis"
%%% End:

\chapter{The \fshark{} language}
\label{chap:fsharklanguage}

% introduction

\begin{figure}
  \centering
  \begin{tabular}{lclr}
    $t$ & $::=$ & $\texttt{int}$ & (Integers) \\
        & $|$   & $\texttt{float}$ & (Floats) \\
        & $|$   & $\texttt{bool}$ & (Booleans) \\
        & $|$   & $(t_0 \times \ldots \times t_n)$ & (Tuples) \\
        & $|$   & $\{\texttt{id}_0:t_0 ; \ldots ; \texttt{id}_n:t_n\}$ & (Records) \\
        & $|$   & $t$ array & (Arrays) \\
    \\

    $k$ & $::=$ & $n$ & (Integer) \\
        & $|$   & $f$ & (Float) \\
        & $|$   & $b$ & (Boolean) \\
        & $|$   & $(k_0 , \ldots , k_n)$ & (Tuple) \\
        & $|$   & $\{\texttt{id}_0=k_0 ; \ldots ; \texttt{id}_n=k_n\}$ & (Record) \\
        & $|$   & $[k_0 ; \ldots ; k_n]$ & (Array) \\
    \\

    $p$ & $::=$ & $\texttt{id}$ & (Name pattern) \\
        & $|$   & $(p_0, \ldots, p_n)$ & (Tuple pattern) \\
  \end{tabular}
  \caption{The FShark syntax}
\end{figure}

\begin{figure}
  \centering
  \begin{tabular}{lclr}
    $e$ & $::=$ & $k$ & Constant \\
        & $|$   & $v$ & Variable \\
        & $|$   & $(k_0 , \ldots , k_n)$ & (Tuple expression) \\
        & $|$   & $\{\texttt{id}_0=k_0 ; \ldots ; \texttt{id}_n=k_n\}$ & (Record expression) \\
        & $|$   & $[k_0 ; \ldots ; k_n]$ & (Array expression) \\
        & $|$   & $e_1 \odot e_2$ & (Binary operator) \\
        & $|$   & $-e$ & (Prefix minus) \\
        & $|$   & \texttt{not} $e$ & (Logical negation) \\
        & $|$   & \texttt{if} $e_1$ \texttt{then} $e_2$ \texttt{else} $e_3$ & (Branching) \\
        & $|$   & $v.[e_0] \ldots .[e_n]$ & (Array indexing) \\
        & $|$   & $v$.\texttt{id} & (Record indexing) \\
        & $|$   & $v_0.v_1$ & (Module indexing) \\
        & $|$   & \texttt{let} $p = e_1$ \texttt{in} $e_2$ & (Pattern binding) \\
        & $|$   & $v$ $e_0$ $\ldots$ $e_n$ & (Function call) \\
    \\
    $entry$ & $::=$ & \texttt{[<FSharkEntry>]} & \\
            & $|$   & $\epsilon$ & \\
    \\
    $fun$ & $::=$ & $entry\ \texttt{let}\ v\ (v_1 : t_1)\ \ldots\ (v_n : t_n) : t = e$ & \\
    $typealias$ & $::=$ & $\texttt{type}\ v\ = t $& \\
    $module$ & $::=$ & $\texttt{module}\ v = prog'\ progs'$ & \\
    \\
    $prog$ & $::=$ & $module\ prog$ & \\
           & $|$   & $prog'\ prog$  & \\
           & $|$   & $\epsilon$     & \\

    $prog'$ & $::=$ & $typealias$   & \\
            & $|$   & $fun$ & \\

    $progs'$ & $::=$ & $prog'\ progs'$   & \\
             & $|$   & $\epsilon$ & \\
  \end{tabular}
  \caption{The FShark syntax, expressions}
\end{figure}

\begin{figure}
  \centering
  \begin{tabular}{lclr}
    $e$ & $::=$ & $k$ & Constant \\
        & $|$   & $v$ & Variable \\
        & $|$   & $(k_0 , \ldots , k_n)$ & (Tuple expression) \\
        & $|$   & $\{\texttt{id}_0=k_0 ; \ldots ; \texttt{id}_n=k_n\}$ & (Record expression) \\
        & $|$   & $[k_0 ; \ldots ; k_n]$ & (Array expression) \\
        & $|$   & $e_1 \odot e_2$ & (Binary operator) \\
        & $|$   & $-e$ & (Prefix minus) \\
        & $|$   & \texttt{not} $e$ & (Logical negation) \\
  \end{tabular}
  \caption{f binary operators}
\end{figure}

\begin{figure}
  \centering
  \begin{tabular}{lclr}
    $e$ & $::=$ & $k$ & Constant \\
        & $|$   & $v$ & Variable \\
        & $|$   & $(k_0 , \ldots , k_n)$ & (Tuple expression) \\
        & $|$   & $\{\texttt{id}_0=k_0 ; \ldots ; \texttt{id}_n=k_n\}$ & (Record expression) \\
        & $|$   & $[k_0 ; \ldots ; k_n]$ & (Array expression) \\
        & $|$   & $e_1 \odot e_2$ & (Binary operator) \\
        & $|$   & $-e$ & (Prefix minus) \\
        & $|$   & \texttt{not} $e$ & (Logical negation) \\
  \end{tabular}
  \caption{FShark SOACs}
\end{figure}

LANGUAGE REFERNCE
\section{The supported F\# subset for FShark}
A standard F\# program automatically includes the \texttt{Microsoft.FSharp.Core}
namespace, which contains the \texttt{Core.Operators} module.
As \texttt{Core.Operators} contains all the basic operators and standard
functions for, this is where the F\# subset suitable for FShark compilation has
been picked out.

\subsection{F\# operators available in FShark}
Infix operators:\\
\begin{tabular}{|c|r|}
  \hline 
  \texttt{+} & Addition \\ \hline
  \texttt{-} & Subtraction \\ \hline
  \texttt{*} & Multiplication \\ \hline
  \texttt{/} & Division \\ \hline
  \texttt{\%} & Modulo operation \\ \hline
  \texttt{**} & Exponentiation \\ \hline
  \texttt{\&\&} & Boolean \texttt{and} \\ \hline
  \texttt{||} & Boolean \texttt{or} \\ \hline

  \texttt{<} & Less \\ \hline
  \texttt{<=} & Less-or-equal \\ \hline
  \texttt{>} & Greater \\ \hline
  \texttt{>=} & Greater-or-equal \\ \hline
  \texttt{=} & Is-Equal \\ \hline
  \texttt{<>} & Is-Not-Equal \\ \hline
  \texttt{<|} & Left-apply: $e_1 \texttt{ <| } e_2$ $\equiv e_1 (e_2)$ \\ \hline
  \texttt{|>} & Right-apply: $e_1 \texttt{ |> } e_2$ $\equiv e_2 (e_1)$ \\ \hline
\end{tabular}
\\
\\
Special operators:\\
\begin{tabular}{|c|r|}
  \hline 
  \texttt{[$e_0$...$e_1$]} & Generate an array of numbers in the interval $[e_0, e_1]$. \\ \hline
\end{tabular}



\subsection{F\# standard functions available in FShark}
\begin{description}
\item[\texttt{id}]\hfill\\
  The identity function
\end{description}

% \include{chapters/3_prelude}
% \include{chapters/4_prelude}
\chapter{The FShark Compiler and Wrapper}
\section*{Introduction}
\label{sec:fsharkcompiler}
Parsing and building a regular F\# program is trivial when using official build tools like
\texttt{msbuild} or \texttt{fsharpc}.
But in the case of FShark, we are not interested in the final result from the
F\# compiler, but merely its half-finished product.

As the F\# Software Foundation offers the official F\# Compiler as a freely
available NuGet package for F\# projects, we can use this package
\texttt{FSharp.Compiler.Services} to parse the entire input FShark program and
give us a Typed Abstract Syntax Tree of the FSharp expressions therein.

FIGURE HERE OF THE USUAL FSHARP COMPILATION 

As the F\# Software Foundation actively encourages developers to create projects
using the F\# compiler library, they have published the collected F\# compiler
as a NuGet package, alongside a tutorial\ref{fsharptutorial}on the usage of the
various compiler parts.

For FShark, the Compiler Services package is used to compile a Typed Abstract
Syntax Tree from a valid FShark source code file, which we then
convert into- and print as a valid Futhark program.
The Typed Abstract Syntax Tree is merely an AST that already has tagged all the
contained expressions with their respective types.

We'll start with a detailed explanation of the FShark Compiler Pipeline.

\subsection{The FShark Compiler Pipeline in practice}
To examine the compiler pipeline in action, we'll go through the motions with
the small example program displayed in figure \ref{fig:fsharkusageexample}.
We begin by constructing an instance of the FSharkWrapper. It has the following
mandatory arguments:

\begin{description}
\item[\texttt{libName}]\hfill\\
  This is the library name for the FShark program. In the final Futhark
  \texttt{.cs} and \texttt{.dll} files, the main class will have the same name
  as the \texttt{libName}. This doesn't really matter if FShark is just used
  as a JIT compiler, but it's good to have a proper name if the user only wants
  to use the compiler parts of FShark.

\item[\texttt{tmpRoot}]\hfill\\
  The FShark compiler works in its own temporary directory. This argument must
  point to a directory where F\# can write files and execute subprocesses
  (Futhark- and C\# compilers) which also has to write files.
  
\item[\texttt{clooPath} and \texttt{monoOptionsPath}]\hfill\\
  The C\# compiler needs the \texttt{Cloo}- and the \texttt{Mono Options}
  libraries available for the compilation, and the finished FShark .dll file
  also needs these two libraries available. To ensure their availability, the
  FSharkWrapper requires these paths at the beginning of the process, so it can
  pass them on later in the process.

\item[\texttt{preludePath}]\hfill\\
  The FShark compiler needs the FShark prelude available to compile FShark
  programs. 

\item[\texttt{openCL}]\hfill\\
  Although Futhark (and therefore FShark) is most effective on OpenCL-enabled
  computers, the benchmarks in \ref{sec:benchmarks} still show a significant
  speed increase for non-OpenCL Futhark over native F\# code.
  Therefore, FShark is also available for non-OpenCL users. Use this flag to
  tell FShark whether Futhark should compile C\# with or without OpenCL.
  
\item[\texttt{unsafe}]\hfill\\
  For some Futhark programs, the Futhark compiler itself is unable to tell
  whether certain array operations or SOAC usages are safe, and will stop the
  compilation, even though the code should (and does) indeed work.
  To enable these unsafe operations, pass a \texttt{true} flag to the compiler.
\end{description}

Now, we can pass a source file to the FShark wrapper, compile\footnote{See
  subsection \ref{subsec:fsharkwrappercompiles}} it and load it into the FShark wrapper object.

To use the compiled FShark function, we must first wrap our designated input in
an \texttt{obj array}. In this case, our chosen FShark function takes one
argument, an \texttt{int array}. We define this array, and construct an argument
array containing this single element. If the FShark function takes two
arguments, we define an input \texttt{obj array} with two elements, and so
forth.
It is important to declare the input array as an \texttt{obj array}. Otherwise,
F\#s own type checker might very well faultily infer the input array as
something else. In this particular case, \texttt{input} would've been inferred
as being an \texttt{int array array}, until we declared its type specifically.

We then invoke the desired function through the wrapper. As all
reflection-invoked functions return a value of type \texttt{obj}, we need to
downcast this object manually.
In this example, we use F\#s downcast operator \texttt{(:?>)} to declare the
return value as an \texttt{int array}. The actual return type is always the same as the
return type declared in the source FShark file.

\subsection{When FShark Wrapper Compiles}
\label{sec:fsharkwrappercompiles}
The general way to compile and load an FShark program into the FShark Wrapper,
is by adding FShark source files to the wrapper object by calling the
\texttt{AddSourceFile} method, and followingly calling the \texttt{CompileAndLoad}
method. Although the FShark wrapper also offers other methods of loading and
compilation, this is the primary one, as it initiates the entire FShark
compilation pipeline.

When calling \texttt{CompileAndLoad}, the supplied FShark source files are
concatenated into one long source file, and written to a temporary location.
An FSharpChecker is then initialized, so we can parse and type check the
concatenated source code. The FSharpChecker is a class exported by the FSharp
Compiler Services, and is a class that lets developers use part of the F\#
compilation pipeline at runtime.

We supply the FSharpChecker with the path to our precompiled FSharkPrelude
assembly, and then call its \texttt{ParseAndCheckProject} method on to receive
an assembly value, which contains the complete Typed Abstract Syntax Tree of our
FShark program, in the form of an \texttt{FSharpImplementationFileDeclaration}.

If the FShark developer followed the guidelines to write a well-formed FShark
module, the main declaration of the program, the
\texttt{FSharpImplementationFileDeclaration}, should contain a single
\texttt{FSharpEntity}, which in turn contains all the remaining declarations in
the program.

\subsubsection{The declaration types within F\#'s Typed AST}
The \texttt{FSharpImplementationFileDeclaration} type has three union cases.
\begin{description}
\item[\texttt{InitAction of FSharpExpr}] \hfill\\
  \texttt{InitAction}s are \fsharpexpr{}s that are executed at the
  initialization of the containing entity. These are not supported in FShark.

\item[\texttt{Entity of FSharpEntity * FSharpImplementationFileDeclaration list}]\hfill\\
  An \texttt{Entity} is the declaration of a type or a module. In the case of
  FShark, three different kinds of entities are supported:
  \begin{description}
  \item[FSharpRecords] are standard record types, and can be translated to
    Futhark records with ease.
    This entity has an empty \texttt{FSharpImplementationFileDeclaration list}.
  \item[FSharpAbbreviations] are type abbreviations, and are easily translated
    into Futhark type aliases.
    This entity has an empty \texttt{FSharpImplementationFileDeclaration list}.
  \item[FSharpModules] are named modules which contains subdeclarations.
    In this case, we retrieve the subdeclarations from the \texttt{FSharpImplementationFileDeclaration list}.
    The FShark compiler supports building FShark modules, but current
    limitations demands that modules are flattened when compiled to Futhark.
    This also means that function name prefixes in function calls are stripped
    when compiled to Futhark.
  \end{description}
\item[\texttt{MemberOrFunctionOrValue of \\ FSharpMemberOrFunctionOrValue *
    FSharpMemberOrFunctionOrValue list list * FSharpExpr}]\hfill\\
  F\# doesn't differ between functions and values, which means that a function
  is merely a value with arguments.
  A pattern matched \texttt{MemberOrFunctionOrValue} value has the form
  \texttt{MemberOrFunctionOrValue (v, args, exp)}, where \texttt{v} contains the
  name and the type of the variable.
  If the \texttt{args} list is empty, \texttt{v} is simply a variable. If not,
  \texttt{v} is a function. \texttt{exp} contains the \fsharpexpr{} that
  \texttt{v} is bound to. An \fsharpexpr{} can be anything from a numeric
  constant to a very long function body.
\end{description}

\subsection{FSharp-to-FSharkIL rules}
INTRODUCTION HERE
For these translations, we will disregard that all \fsharpexpr{}s are union
cases of the F\# data type \texttt{BasicPatterns}.

\begin{figure}
  \centering
\begin{tabular}{@{}l c l c l c l}% to \linewidth {l c X}
  $\evals{Int8}$ & $=$ & $\lit{Prim Int FInt8} $ & ~~~~~ & 
  $\evals{Int16}$ & $=$ & $\lit{Prim Int FInt16}$
  \\
  $\evals{Int32}$ & $=$ & $\lit{Prim Int FInt32} $ & ~~~~~ & 
  $\evals{Int64}$ & $=$ & $\lit{Prim Int FInt64} $
  \\
  $\evals{UInt8}$ & $=$ & $\lit{Prim UInt FUInt8} $ & ~~~~~ & 
  $\evals{UInt16}$ & $=$ & $\lit{Prim UInt FUInt16} $ 
  \\
  $\evals{UInt32}$ & $=$ & $\lit{Prim UInt FUInt32} $ & ~~~~~ & 
  $\evals{UInt64}$ & $=$ & $\lit{Prim UInt FUInt64} $ 
  \\
  $\evals{Single}$ & $=$ & $\lit{Prim Float FSingle} $ & ~~~~~ & 
  $\evals{Double}$ & $=$ & $\lit{Prim Float FDouble} $ 
  \\
  $\evals{\tau []}$ & $=$ & $\lit{FSharkArray }\evals{\tau}$ & ~~~~~ & 
  $\evals{(\tau_0 \times \ldots \times \tau_n)}$ & $=$ & $\lit{FSharkTuple}~\evals{\tau_0},\ldots,\evals{\tau_n}$
  \\
\end{tabular}

INSERT NOTE ON RULE FOR TUPLE ('a [] * long [])

\caption{F\# (.NET) types to FSharkIL types}
\end{figure}


\begin{figure}
  \centering
\begin{tabular}{@{}l c l c l c l c l c l}% to \linewidth {l c X}
  $\evals{Const(obj, \tau)}$ & $=$ & $\lit{Const}(obj,}\evals{\tau}}\lit{)}$ & ~~~~~ & 
  $\evals{i}$ & $=$ & $i$ & ~~~~~ & 
  $\evals{f}$ & $=$ & $f$ \\
  $\evals{c}$ & $=$ & $\id{ascii}(c)$ & ~~~~~ & 
  $\evals{\lit{tt}}$ & $=$ & \lit{true} & ~~~~~ & 
  $\evals{\lit{ff}}$ & $=$ & \lit{false} \\ ~ \\
\end{tabular}
  \caption{FShark SOACs}
\end{figure}





\begin{figure}
  \centering
    \begin{minted}[linenos,breaklines]{fsharp}
module FSharkExample
open FShark.Main

[<EntryPoint>]
let main argv =
  let wrapper = 
    new FSharkWrapper(
      libName="ExampleModule",
      tmpRoot="/home/mikkel/FShark",
      clooPath="/home/mikkel/Cloo.clSharp.dll",
      monoOptionsPath="/home/mikkel/Mono.Options.dll",
      preludePath= "/home/mikkel/Documents/fshark/FSharkPrelude/bin/Debug/FSharkPrelude.dll",
      openCL=true,
      unsafe=true
      )

  wrapper.AddSourceFile "../../srcs/ExampleModule.fs"
  wrapper.CompileAndLoad
  let xs = [|1;2;3;4|]
  let input = [|xs|] : obj array
  let xs' = wrapper.InvokeFunction "MapPlusTwo" input :?> int array
  printfn "Mapping (+2) over %A gives us %A" xs xs'
  0
    \end{minted}
  \caption{An F\# program using FShark}
  \label{fig:fsharkusageexample}
\end{figure}

\begin{figure}
  \centering
  \begin{minted}[xleftmargin=5pt,linenos]{fsharp}
    module ExampleModule
    open FSharkPrelude

    module SomeValues =
      let Four : int = 4

      let SomePlus (x : int) (y : int) : int = x + y

    [<FSharkEntry>]
    let TimesTwo (x : int) : int =
      SomeValues.SomePlus x x
  
    [<FSharkEntry>]
    let MapPlusTwo (xs : int array) : int array =
      Map ((+)2) xs

    let PlusSeven (x : int) : int =
      SomeValues.SomePlus x 7
  \end{minted}
  \caption{A valid FShark program}
  \label{fig:validfsharkprogram}
\end{figure}

\begin{figure}
  \centering
  \begin{minted}{text}
    let Four : i32 = 4i32
    let SomePlus (x : i32) (y : i32) : i32 =
      ((x i32.+ y))
    entry TimesTwo (x : i32) : i32 =
      unsafe SomePlus(x) (x)
    entry MapPlusTwo (xs : []i32) : []i32 =
      unsafe map (let x = 2i32 in
                  (\(y : i32) -> ((x i32.+ y)))) (xs)
    let PlusSeven (x : i32) : i32 =
      SomePlus(x) (7i32)
      \end{minted}
  \caption{A valid FShark program, compiled to Futhark}
  \label{fig:validfsharkprogramresult}
\end{figure}

  
In figure \ref{fig:validfsharkprogram} we see a small but valid FShark program. It
reads like a regular F\# program, but contains the three vital parts that makes
it usable as an FShark program.
\begin{itemize}
\item The module declaration on the first line declares that the following code
  is inside a module. In this case, we are declaring the module
  \texttt{ExampleModule}, although we could use any valid F\# module name.
  As shown in figure \ref{fig:validfsharkprogramresult}, the top module
  declaration falls away during compilation, so only the top module contents are
  left.

\item This \texttt{open} statement ensures that the F\# Compiler Services has
  access to the FSharkPrelude during the compilation. It is possible to write an
  FShark program which doesn't use the FSharkPrelude, but this removes access to
  the SOACs that we use to write our data parallel programs.

\item The \texttt{[<FSharkEntry>]} attributed function \texttt{TimesTwo} ensures
  that the resulting Futhark library from the FShark compiler has at least one
  entry point function.
  Without any entry point functions, we won't have any functions in the final
  compiled FShark program.
\end{itemize}

In figure \ref{fig:validfsharkprogramresult} we see the resulting Futhark program.
For now, we will ignore the transformations that have happened, except for two
things: The \texttt{Map} function (called from FSharkPrelude) has been rewritten
as the plain Futhark SOAC \texttt{map} in lowercase, and the module SomeValues has been
flattened (see sec \ref{futurework:modules} for future plans.)

This Futhark program is then stored in a temporary location in the user's file
system, and compiled into as a library, using Futhark's C\# compiler, either
with or without OpenCL support. Finally after this compilation, we can invoke
the resulting .dll file from within the FShark-using F\# program.

\subsection{Building FShark from the Typed AST}
\label{sec:fsharkcompilerrules}
FShark supports a subset of the F\# language, which also means that only a
subset of F\#'s FSharpExpr

Only the supported FSharpExpr's has been listed here, but the full range of
FSharpExpr's are available on \cite{typedtree}.

%%% Local Variables:
%%% mode: latex
%%% TeX-master: "../thesis"
%%% End:
% \include{chapters/6_prelude}

% \include{chapters/7_prelude}
% \include{chapters/8_prelude}
% \include{chapters/9_prelude}
% \include{chapters/10_prelude}
% \include{chapters/11_prelude}
% \include{chapters/12_prelude}
% \include{chapters/a_appendices}
\chapter{Array handling in FShark}
% introduction
\fsharp{} is a functional programming language on top of the .NET framework, which
means that it's primitive types like \texttt{int, float} and \texttt{list} all
correspond to already existing classes in the .NET framework. For example,
\fsharp{}'s \texttt{int} is an alias for .NET's \texttt{System.Int32} and
\texttt{float} is an alias for \texttt{System.Double}.

Therefore, we also find corresponding .NET classes for both \fsharp{}
\texttt{list}s and \texttt{array}s. \texttt{list}s are
\texttt{FSharp.Collections.FSharpList}s, and \texttt{array}s are
\texttt{System.Array}. (Note that \texttt{FSharpList}s is available from any
.NET framework language, as long as the corresponding assembly is referenced).

Although it is common to use lists in functional programs, the \fsharp{} subset
covered by FShark does not include lists -- In Futhark, and therefore also
FShark, our main goal is not handling list elements one after another, but
rather parallelizing computations across entire arrays of data simultaneously.

The \texttt{FSharp.Collections.FSharpList} is implemented as a singly-linked
list. SOACs called on singly-linked lists are inherently unparallelizable, as
the SOACs must traverse the list sequentially.
For example, calling $\texttt{map} f$ on a singly linked list \texttt{(x::xs)}
means computing $f \texttt{x}$ and inserting the result into $\texttt{map} f
\texttt{xs}$ recursively. We can do some parallel computations for these kinds
of SOACs, i.e. by making the main thread traverse the list and spawn
a thread for each element computation. However we will still have suboptimal
memory access performance, as the elements in the singly linked list doesn't
have any guarantees regarding their location in RAM, which means we are going to
perform many more memory loads compared to if we were performing calculations on
elements in a sequentially stored array elements in RAM.


% a basic description of lists in F# and arrays in F#

% short description of multidims in F#
% description of jagged arrays in F#
% description of array handling in Futhark 

% a description of SOACs in FSharkPrelude, analysis of various time complexities
% for FSharkPrelude SOACs

%% handling irregular arrays at runtime when using FSharkPrelude.

% Wrapping and unwrapping when invoking functions:
%% Why do we need to completely flatten arrays before passing into Futhark
%% module?
%%% Array flattening algorithm in pseudocode
%%% Cost of algorithm


% Reshaping flat array into jagged array
%% Description of algorithm
%% Cost of algorithm

% performance??

% alternative solutions: FSharkArrays

%%% Local Variables:
%%% mode: latex
%%% TeX-master: "../thesis"
%%% End:
\chapter{\fshark{}s interoperability between \fsharp and Futhark (\csharp{})}
FShark stands on three legs: The FShark compiler, the Futhark C\# code generator, and
the FSharkWrapper.
The compiler is responsible for compiling FShark source code into Futhark
source code, and the C\# code generator takes the result Futhark source code,
and compiles OpenCL powered C\# libraries, which can be imported directly back
into F\#.

It is of course possible to use the compiler and the code generator as
individual modules, but for this project, the FSharkWrapper has been designed to
let users use FShark without having to understand any of the underlying
pipeline.

To illustrate this; take a look at figure \ref{fig:fsharkusingwrapper}. In the
first line, the user initializes the FSharkWrapper with the arguments necessary
to use the wrapper itself. In the second line, the user adds a source file to
the wrapper by it's path.
In the third line, the user tells the wrapper to run the compilation pipeline.
Assuming that the compilation goes well, the user can then invoke some function
from the FShark program in line four.

Here, calling the \texttt{CompileAndLoad()} function triggers the entire \fshark{}
pipeline as described in \ref{pipeline}, and does then have a function available
for the user to call afterwards.

However, as this is the default way of using \fshark{}, we are currently calling
\texttt{CompileAndLoad()} every time we use the \fshark{} program.
This is happening even though we only need the final compiled C\# assembly to
load back into \fsharp{} at runtime. 

Everytime we run the FShark compiler pipeline, we are therefore also
\begin{enumerate}
\item parsing, typechecking and generating a TAST from the \fshark{} code,
  using FSharp's compiler.
\item generating Futhark source code from the FSharp TAST
\item Writing the Futhark source code to disk
\item running the Futhark compiler and C\# code generator on the Futhark source
  code
\item running the mono C\# compiler on the resulting C\# source code
\end{enumerate}

For two selected benchmarks we have the following times
% Numbers for nbody.fut:
%  FShark parsing took 1850345 ms
%  FSharpDecls to FSharkIL took 76660 ms
%  FSharkIL to Futhark source code took 43471 ms
%  Compiling the Futhark module into .cs source code took 1555105 ms
%  Compiling .cs module took 999251 ms
%  Loading compiled .cs assembly using reflection took 101601 ms
%  The entire FShark compilation pipeline took 4634460 ms
%  Average invokation time was 104227 ms
%  Invoking main took 107597 ms

% LocVolCalib.fut small
%FShark parsing took 217984 ms
%FSharpDecls to FSharkIL took 19129 ms
%FSharkIL to Futhark source code took 98949 ms
%Compiling the Futhark module into .cs source code took 8037165 ms
%Compiling .cs module took 1422527 ms
%Loading compiled .cs assembly using reflection took 81370 ms
%The entire FShark compilation pipeline took 9877907 ms
%Average invokation time was 219150 ms
%
% locvolcalib medium
% FShark parsing took 221395 ms
% FSharpDecls to FSharkIL took 21487 ms
% FSharkIL to Futhark source code took 107558 ms
% Compiling the Futhark module into .cs source code took 7739211 ms
% Compiling .cs module took 1467667 ms
% Loading compiled .cs assembly using reflection took 77850 ms
% The entire FShark compilation pipeline took 9635733 ms
% Average invokation time was 333960 ms
% 
% locvolcalib large
% FShark parsing took 229771 ms
% FSharpDecls to FSharkIL took 20973 ms
% FSharkIL to Futhark source code took 111219 ms
% Compiling the Futhark module into .cs source code took 7452111 ms
% Compiling .cs module took 1422789 ms
% Loading compiled .cs assembly using reflection took 79751 ms
% The entire FShark compilation pipeline took 9317354 ms
% Average invokation time was 5863264 ms

% nbody from assembly:
% Opening class from assembly took 61841 ms
% Running nbody from assembly took 117373 ms

% locvolcalib
% small
% Opening class from assembly took 147341 ms
% Running locvol small from assembly took 210698 ms
% medium
% Opening class from assembly took 61771 ms
% Running locvol medium from assembly took 314444 ms
% large
% Opening class from assembly took 71198 ms




% make a chart that shows the time consumption of a regular FShark execution
% (for LocVolCalib and nbody) from start to finish, and what the time is spent
% on 

% make the same chart but where the assembly is referenced at compile time,
% which means that the FShark program is just used as a compiler.

% for both tests, we use ArrayToFlatArray, as we have still settled on normal
% arrays as the array.

\subsection{Pros and cons of the current design}
As there are demonstratively great performance gains to be won by only using the
compiler part of the FShark pipeline, it is worth discussing whether the
rest of the FShark pipeline should remain.

Besides eliminating the entire compilation operation at every \fshark{}
execution, a compiler-only approach to the \fshark{} compiler would give us the
following advantages:
\begin{itemize}
\item \textbf{Standalone-modules first:} As the compiler is now only
  used once, the resulting Futhark assembly is readily importable in any .NET
  project, as long as the required Mono libraries are also available.
  This goes not only for the user who just compiled the assembly, but also for
  any other user who has acquired the necessary Mono libraries. This means that
  the \fshark{} developer can use and share the \fshark{} assemblies with
  colleagues and coworkers like any other sharable .NET library,
  as this is indeed what a compiled Futhark
  \csharp{} library amounts to.

  Corollarily; this \fshark{} design would make \fshark{} is useful for generating
  high-performing .NET libraries. (Although one could write such libraries in
  Futhark instead of \fshark{}.)

\item \textbf{Static typing of \fshark{} module:} The current runtime-only
  approach means that the user must rely on reflection to call \fshark{}
  functions.
  In this situation, all modern IDE comforts like autocompletion, and especially
  static type checking and inference falls away.
  For the following example
  %% let foo (x : int) ((y,z) : (int * int)) : (int * int) 
  , the current design demands that we first wrap our arguments in an
  \texttt{obj array}, before invoking the function \texttt{foo}. Furthermore, we
  must also downcast the result using \fsharp{}s downcasting operator \texttt{:?>}.
  Because we are upcasting our arguments to an \texttt{obj array}, we can
  actually pass any (correctly casted) array as an argument to our
  reflection-invoked function, without triggering any type errors at compile
  time.
  The same goes for the downcasted result from the function. We can cast the
  result as whatever type we like, and not run into any trouble until we finally
  run the compiled program.
  % write the example.
  However, if we use \fshark{} to generate assemblies instead, we now have all
  the type information available at compile time. Our compiler will block us
  from compiling the program by giving us useful type errors. Last but not
  least, we can remove all the casting operations that are littering the
  program.
  % other example
  
\item \textbf{Getting rid of, or at least trimming down, the FSharkWrapper:} 
\end{itemize}

However, the current design of \fshark{} also has some advantages that follows
automatically from the design.

\begin{itemize}
\item \textbf{Rapid \fshark{} code development:}
  Currently, it is recommended that any \fshark{} code for a project is also
  built as part of the project.
  By including the \fshark{} file in the original project's source list, we can
  call the \fshark{} module natively, without running the \fshark{} compiler, to
  prototype and debug the \fshark{} code directly in our IDE, before we switch
  to using the compiled version of the \fshark{} code.
  % example of developing
  % 1: picture of placeholder function from Fshark, and fsharp module calling this
  % 2: picture of placeholder function filled out, and fsharp module calling
  % this
  % 3: picture of placeholder function filled out, and fsharp module calling
  % this through fshark

\item \textbf{En mere}
\end{itemize}


% Discuss the pros and cons of the current design:
%% Pros:
%% 1) rapid development. FShark development can take place within the same IDE
%% and project as the FShark-using program.
%% 2) Ease of use: FShark can be used through a few imports and 

%% Cons:
%% 1) Probably the time consumption
%% 2) The need for the FShark wrapper. Because the runtime-only design demands
%% using reflection, the reflection behaviour has been encapsulated in a bulky
%% wrapper object.
\section{The future of \fshark{} interoperability}
With these considerations in mind, my future work on \fshark{}s interoperability
consists of reducing FSharkWrapper in size, so it only takes an \fshark{} source path
  and a .NET assembly outpath as inputs, and does nothing more than
  orchestrating the \fshark{}-, the Futhark- and the \csharp compiler.
  The current design is too complex, largely from supporting too many superflous
  features like concatenating multiple sources, and so on.

I will also be researching the optimal way to keeping the \fshark{} module development as
  close to the rest of the \fshark{}-using project as possible, without %% MERE
  %% HERE
  
The current design enables direct prototyping, which must of course be kept
in later versions of \fshark{}.

%% Another idea: keep annoying wrapper, but let it skip the compiling step.

% \fshark{} as a library generator

%%% Local Variables:
%%% mode: latex
%%% TeX-master: "../thesis"
%%% End:



\end{document}


%%% Local Variables:
%%% coding: utf-8
%%% mode: latex
%%% TeX-command-extra-options: "-shell-escape"
%%% End: