\documentclass{article}
\usepackage{amsmath} % Required for some math elements 
\usepackage{graphicx} % Required for some math elements 

\newcommand{\fshark}{\texttt{F\#thark}}
\begin{document}
\noindent
\textbf{
    \fshark \--- Effortless Futhark programming in F# projects \\
    Mikkel Storgaard Knudsen\\
    Project description    
}

\section{Problem background}

As CPU performance on per-core basis has slowed down significantly since
mid 2000, the trend in modern architectures design has been to enable 
Moore's low by exponentially scaling the hardware parallelism. 
For example multi-cores have commonly up to 32 cores, whilst modern 
GPGPUs support thousands of cores.
These highly-parallel hardware however, complicates the job of the 
programmer who now has to reason and express not only application 
parallelism but also to optimise access patterns for the capricious 
many-core hardware (e.g., optimising locality of reference).

1. DSLs for programming productivity: >
The parallel assembly languages of our time OpenCL and CUDA are tedious.
This has lead to the design of various data-parallel DSLs covering 
applications from various fields ...
Futhark is a general-purpose, hardware-neutral, purely-functional data-parallel 
language (and optimising-compiler infrastructure) aimed at efficient execution 
on many-core hardware such as GPGPUs (ref 1). 

2. An common critique against the approach of using data-parallel languages (DSL) 
for harnessing the power of modern hardware, has been that in an enterprise 
setting it can be seen as frivolous to embark on exploratory project with 
DSLs whose potential and lifetime is questionable.  In other words, if the
application code is not written in a mainstream language than it is ``esoteric
and unmaintainable'', and as such it is an unacceptable approach.

3. This project plans to investigate the feasibility of a deep-interoperability
solution between one such data-parallel DSL, namely Futhark, and one such
mainstream productivity-oriented environment, namely F\#. 
\begin{itemize}
    \item[1] an F\# code generator, that would allow ``small'' computational
        kernels to be written in Futhark and easily integrated with the
        rest of the F\# application. In some sense, efficient many-core
        execution of such computational kernels promotes language interoperability
        because performance-wise it matters little if the CPU-orchestration code
        is written in a slower or faster (mainstream) language.
    \item[2] The more challenging part of the project refers to identifying
        a subset of F\# that can be (always and efficiently) translated to 
        Futhark---think Featherweight Java for example.
        This step allows to effectively hide Futhark from the user toolchain:
        programs can be developed entirely in F\#, using the standard 
        programming-productivity facilities of F\# (F\# SDK)---such as debugger,
        interactive consoles, profilers---and at the very end (after unit testing and all that)
        programs conforming to the identified subset can be compiled to Futhark
        and from there to efficient GPGPU code.
\end{itemize}




Although the Futhark compiler creates heavily optimized OpenCL programs from
Futhark source code, the developer has to write his or her program in Futhark.
It is up for discussion how high the Futhark barrier of entry actually is, 
but in any case it is not realistic to expect all developers to learn an entire 
(functional) programming language, just to gain some expected 
performance increase.

\fshark seeks to lower this barrier of entry significantly.
By using \fshark, the F\# merely has to insert the \fshark attribute to their
already existing code. \fshark will then automatically obtain Futhark optimized
OpenCL source code from the F\# code, inject this into the F\# code, and compile
it into an F\# program as usual.


In short; \fshark lets F\# developers harness the optimizations of Futhark
programming by adding a simple [Fshark] attribute to their code.

As functional programs has a reputation of being slower than equivalent object
oriented programs, I will not just investigate the performance increase between
standard- and \fshark-enabled F\# programs, but also between \fshark-enabled F#
programs and OpenCL C\# programs.

\section{Goals}
The overall goal of this project is to
\begin{itemize}
    \item develop a C\# code generator for Futhark's CPU-orchestration code,
            together with a method of inserting Futhark generated OpenCL code into
            an existing F\# code base.

    \item identify a subset of F\# that can be effectively translated to Futhark,
            develop rewrite rules for transpiling the identified subset of F\# to 
            Futhark


    \item demonstrate the F\#-Futhark interoperability solution by analysing 
            a set of big-compute benchmarks.
    % between F\#- and \fshark programs, and between
    %C\# OpenCL- and \fshark programs.
\end{itemize}

\section{Tasks}
\begin{enumerate}
    \item develop an intermediate language between F\# and Futhark.
    I.e., F\# has I/O constructs that does not have any equivalents within 
    Futhark, which means that an \fshark function must be split into an I/O part
    and a pure part.

    \item develop a transpiler from (a subset of) F\# to this intermediate
    language.

    \item develop a transpiler from this intermediate language to Futhark.

    \item develop an F\# code generator which writes complete compilable F\#
    modules, containing the OpenCL source code returned from the Futhark 
    compiler.

    \item develop a method to replace the \fshark enabled code with a modified 
    version, that contains the code generated in step 4.
    
    \item Compartmentalize all the above steps into a preprocessor
    that is executed before running the usual F\# compiler.
    
    \item show equivalence between the original \fshark code, and the 
    intermediate language code that is being sent to Futhark.

    \item show equivalence between the intermediate language code,
    and the Futhark code it is being translated into.

    \item run and document benchmarks between F\#/\fshark and 
    C\# (OpenCL)/\fshark 

    \item syntax-coloring tool for the F\# subset
\end{enumerate}

\section{Learning objectives}
\begin{itemize}
    \item explain subsets of Futhark and F\# languages, which are representative 
            for the translation.
    \item review various transpiler solutions from literature
    \item 
    \item design, implement and evaluate the complete set of rewrite 
            rules required by the F\#-to-Futhark transpilation.
    \item evaluate the solution on a set of representative benchmarks
    \item argue for the correctness of the transformations
    \item discuss (possibly speculatively, but reasoned) the ease of use of the framework
    \item Design a transpiler that reads code from a mixed paradigm language,
    and splits it into an impure part, and a pure functional part.
    
\end{itemize}

\section*{References}
\end{document}
