\documentclass{article}
\usepackage{amsmath} % Required for some math elements 
\usepackage{graphicx} % Required for some math elements 

\newcommand{\fshark}{\texttt{F\#thark}}
\begin{document}
\noindent
\textbf{
    \fshark \--- Effortless Futhark programming in F# projects \\
    Mikkel Storgaard Knudsen\\
    Project description    
}

\section{Problem background}
Futhark has already shown its efficiency, 
and in many cases supremacy, compared to many established performance benchmarks
for GPU code (ref 1). 

As CPU performance on per-core basis has slowed down very significantly since 
year \ref, continued performance increases in performance-oriented programming 
has been found by writing parallel programs in favour of sequential programs.
Multicore CPUs most commonly have 4 to 32 cores, whilst modern GPGPUs can have
multiple thousands of cores.

Although the Futhark compiler creates heavily optimized OpenCL programs from
Futhark source code, the developer has to write his or her program in Futhark.
It is up for discussion how high the Futhark barrier of entry actually is, 
but in any case it is not realistic to expect all developers to learn an entire 
(functional) programming language, just to gain some expected 
performance increase.

\fshark seeks to lower this barrier of entry significantly.
By using \fshark, the F\# merely has to insert the \fshark attribute to their
already existing code. \fshark will then automatically obtain Futhark optimized
OpenCL source code from the F\# code, inject this into the F\# code, and compile
it into an F\# program as usual.


In short; \fshark lets F\# developers harness the optimizations of Futhark
programming by adding a simple [Fshark] attribute to their code.

As functional programs has a reputation of being slower than equivalent object
oriented programs, I will not just investigate the performance increase between
standard- and \fshark-enabled F\# programs, but also between \fshark-enabled F#
programs and OpenCL C\# programs.

\section{Goals}
The overall goal of this project is to
\begin{itemize}
    \item develop a transpiler from (a subset of) F\# to Futhark

    \item develop a method of inserting Futhark generated OpenCL code into
    an existing F\# code base.

    \item explore benchmarks between F\#- and \fshark programs, and between
    C\# OpenCL- and \fshark programs.
\end{itemize}

\section{Tasks}
\begin{enumerate}
    \item develop an intermediate language between F\# and Futhark.
    I.e., F\# has I/O constructs that does not have any equivalents within 
    Futhark, which means that an \fshark function must be split into an I/O part
    and a pure part.

    \item develop a transpiler from (a subset of) F\# to this intermediate
    language.

    \item develop a transpiler from this intermediate language to Futhark.

    \item develop an F\# code generator which writes complete compilable F\#
    modules, containing the OpenCL source code returned from the Futhark 
    compiler.

    \item develop a method to replace the \fshark enabled code with a modified 
    version, that contains the code generated in step 4.
    
    \item Compartmentalize all the above steps into a preprocessor
    that is executed before running the usual F\# compiler.
    
    \item show equivalence between the original \fshark code, and the 
    intermediate language code that is being sent to Futhark.

    \item show equivalence between the intermediate language code,
    and the Futhark code it is being translated into.

    \item run and document benchmarks between F\#/\fshark and 
    C\# (OpenCL)/\fshark 
\end{enumerate}

\section{Learning objectives}
\begin{itemize}
    \item Design a transpiler that reads code from a mixed paradigm language,
    and splits it into an impure part, and a pure functional part.
    
\end{itemize}

\section*{References}
\end{document}