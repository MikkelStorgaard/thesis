\documentclass{article}
\usepackage{amsmath} % Required for some math elements 
\usepackage{graphicx} % Required for some math elements 

\newcommand{\fshark}[0]{\texttt{F\#thark}}
\begin{document}
\noindent
\textbf{Effortless Futhark programming in F\# projects}\\
\textbf{Mikkel Storgaard Knudsen}
\textbf{Project description}

\section{Problem background}

As CPU performance on per-core basis has slowed down significantly since
mid 2000, the trend in modern architectures design has been to enable 
Moore's low by exponentially scaling the hardware parallelism. 
For example multi-cores have commonly up to 32 cores, whilst modern 
GPGPUs support thousands of cores.
These highly-parallel hardware however, complicates the job of the 
programmer who now has to reason and express not only application 
parallelism but also to optimise access patterns for the capricious 
many-core hardware (e.g., optimising locality of reference).

The parallel assembly languages of our time OpenCL and CUDA are tedious.
This has lead to the design of various data-parallel DSLs covering 
applications from various fields. In example, Diderot\cite{pldi12} is a highly
specialized image analysis language, and TAIL\cite{fhcp16} compiles APL into GPU code.
Futhark is a general-purpose, hardware-neutral, purely-functional data-parallel 
language (and optimising-compiler infrastructure) aimed at efficient execution 
on many-core hardware such as GPGPUs\cite{pldi17}.

An common critique against the approach of using data-parallel languages (DSL) 
for harnessing the power of modern hardware, has been that in an enterprise 
setting it can be seen as frivolous to embark on exploratory project with 
DSLs whose potential and lifetime is questionable.  In other words, if the
application code is not written in a mainstream language than it is ``esoteric
and unmaintainable'', and as such it is an unacceptable approach.

This project plans to investigate the feasibility of a deep-interoperability
solution between one such data-parallel DSL, namely Futhark, and one such
mainstream productivity-oriented environment, namely F\#.

Mainly, this project consists of two parts:
\begin{enumerate}
    \item an F\# code generator, that would allow ``small'' computational
        kernels to be written in Futhark and easily integrated with the
        rest of the F\# application. In some sense, efficient many-core
        execution of such computational kernels promotes language interoperability
        because performance-wise it matters little if the CPU-orchestration code
        is written in a slower or faster (mainstream) language.
    \item The more challenging part of the project refers to identifying
        a subset of F\# that can be (always and efficiently) translated to 
        Futhark---think Featherweight Java for example.
        This step allows to effectively hide Futhark from the user toolchain:
        programs can be developed entirely in F\#, using the standard 
        programming-productivity facilities of F\# (F\# SDK)---such as debugger,
        interactive consoles, profilers---and at the very end (after unit testing and all that)
        programs conforming to the identified subset can be compiled to Futhark
        and from there to efficient GPGPU code.
\end{enumerate}

The final product of this project should let any F\# developer harness the
performance gains of working with Futhark, without modifying their existing
project by more than a single extra line or two in the source code.
Such a non-invasive approach to F\#-Futhark interoperability is neither
``esoteric'' nor ``unmaintainable'', and is indeed something resembling a ``free
lunch'' solution for using Futhark in an F\# environment. 

\section{Goals}
The overall goal of this project is to
\begin{itemize}
    \item develop a C\# code generator for Futhark's CPU-orchestration code,
            together with a method of inserting Futhark generated OpenCL code into
            an existing F\# code base.

    \item identify a subset of F\# that can be effectively translated to Futhark,
            develop rewrite rules for transpiling the identified subset of F\# to 
            Futhark


    \item demonstrate the F\#-Futhark interoperability solution by analysing 
            a set of big-compute benchmarks.

    % between F\#- and \fshark programs, and between
    %C\# OpenCL- and \fshark programs.
\end{itemize}

\section{Tasks}
\begin{enumerate}
    \item develop an intermediate language between F\# and Futhark.
    I.e., F\# has I/O constructs that does not have any equivalents within 
    Futhark, so the intermediate code must split F\# into untranslatable and
    translatable sections, and pass the translatable parts on to Futhark.

    \item develop a C\# code generator which takes intermediate code, and
      Futhark-compiled OpenCL source code, and generates a C\# module that is
      fully callable from the original F\# code.

    \item develop a preprocessor that performs the F\#-to-Futhark (and back
      again) compilations as an automatic step before compiling the resulting
      F\# program using the conventional toolchain.
    
    \item show semantic equivalence between written F\# source code,
      and rewritten intermediate code (and the resulting C\# code.)

    \item run benchmarks and evaluate performance between F\#/F\# w. Futhark, and 
      C\# (OpenCL)/ F\# w. Futhark

    \item develop a syntax-coloring tool for the F\# subset
\end{enumerate}

\section{Learning objectives}
\begin{itemize}
    \item explain subsets of Futhark and F\# languages, which are representative 
            for the translation.
    \item review various transpiler solutions from literature
    \item design, implement and evaluate the complete set of rewrite 
            rules required by the F\#-to-Futhark transpilation.
    \item evaluate the solution on a set of representative benchmarks
    \item argue for the correctness of the transformations
    \item discuss (possibly speculatively, but reasoned) the ease of use of the framework
\end{itemize}

\bibliographystyle{plain}
\bibliography{references}

\end{document}
